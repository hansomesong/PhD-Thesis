\section{Introduction}
\label{sec:introduction}
In previous chapters, the performance of MTC over LPWAN networks is analyzed. Although it is an attractive option to deploy M2M applications over LPWAN networks such as Sigfox or LoRaWAN, the operators have to maintain two separate infrastructures for their clients. Hence, due to the ubiquitous coverage area and mature infrastructure, the accommodation of MTC traffic over existing cellular networks is also attractive. 
Up to this day, the cellular wireless network technology has evolved to the $4^{\text{th}}$ generation, and is now at the gate of $5^{\text{th}}$ generation era. Recall that one requirement for MTC is that the deployed devices are expected to remain operational for decades~\cite{GrowthInM2M2011}. First, it is necessary to know which generation of cellular networks is more suitable for MTC.

The 2G networks, such as GSM technology, are energy efficient and technically a very suitable choice for MTC. After decades development, it is a rather mature technology. With minimum reengineering work, they can well support MTC. However, M2M devices are expected to remain operational for decades while 2G networks are planed to be decommissioned within the next $5-15$ years. For example, AT\&T decided to stop GSM-based service and refarm the spectrum occupied by GSM to LTE in 2017~\cite{att2014}. For a long term view, 2G is not the most appropriate candidate for MTC. 3G is based on spectrum spreading technology and is not energy efficient, thus few research efforts are made to consider how to support MTC over 3G networks. 4G, now with wide deployment and still in its life circle, is a safe and appropriate choice. However, this technology was designed to solve mobile broadband demand for human beings and not for the particular requirements of MTC, which are quite different from human type communication (HTC). To efficiently support MTC, LTE has to be adapted with respect to MTC characteristics. 

In this chapter, we present our efforts to efficiently handle large number of devices over future LTE networks. Inspired from the observation that a considerable partition of MTC traffic exhibit periodicity~\cite{Costa14}, we propose a multiple periodic polling service in LTE radio access network to serve a large number of MTC devices deployed for M2M periodic application, such as smart grid. Our proposal reduces the transmission overhead and thus improves the energy efficiency for MTC devices. It also reduces signaling overload in radio access network by avoiding random access.  
The proposed service can be integrated into LTE radio access network, fully compatible with the standard access mechanism and able to manage a large range of polling periods (typically from \boldmath{$1$} minute to \boldmath{$28$} days). Compared with traditional random access mechanism, numerical results show that applying our proposal one eNodeB (eNB) can easily support up to \boldmath{$15000$} MTC devices without network access collision.  

This chapter is organized as follows: Section~\ref{sec:related-work} talks about related work, Section~\ref{sec:LTE-random-access} gives an overview about the work-flow of MTC device periodic data report via random access method. Section~\ref{sec:polling} presents in detail our M2M-oriented multiple-period polling mechanism. Section~\ref{analysis} conducts a comparative performance analysis between our proposal and conventional LTE random access when handling MTC traffic. Section~\ref{sec:conclusion} is the conclusion part of this chapter.

%existing M2M applications in cellular network can be classified into 4 categories :  time-driven, query-driven, event-driven and hybrid-driven. Time-driven type (also called periodic M2M) actually holds the most large portion of all \cite{Costa14}. 
% 题目里面 明明说的是 提共的是一个service, 这里却说的是 我们研究的是 access and resource allocation...
% 是不是需要 考虑如何说 更好呢