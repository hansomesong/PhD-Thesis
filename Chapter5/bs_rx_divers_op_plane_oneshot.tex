\section{Network Level Performance Without Retransmission}
\label{sec:op_over_infinite_plane}
The analysis about link-level transmission success probability $p_s(r)$ paves the way to analyze the network level packet loss rate $P_{f}$, which is the base to calculate:\begin{inparaenum}[1)]
	\item the maximum normalized load $L_{\text{max}}$ under packet loss rate constraint $P_{f}^{\text{max}}$;
	\item the normalized spatial throughput (per BS) $S = (1-P_{f}) p\lambda_{m}/ \lambda_{b}$
	%TODO should we analyze energy-efficiency?
\end{inparaenum}
We first consider two standard ALOHA transmission approaches in unicast mode (best and nearest BS attach) and then analyze the macro diversity case. For macro diversity case, we also consider two cases: selective ratio and maximum ratio combing.

Note that to keep the tractability of system model, when background noise is taken into account, we let path loss exponent $\gamma=4$. For some cases, background is neglected to obtain closed-form expression.
\subsection{Nearest BS attach method}
\label{sec:nearest_BS_attach_method}
%A device can attach to the best BS which the received power of packet transmission is the strongest if the channel state info is available during the communication.
As a complementary analysis, in this subsection, we assume that the device attaches to the geographically nearest BS then transmits packets. The interest of conducting such an analysis are: give an upper bound of packet loss rate and show an accuracy approximation method when shadowing is present. The closed-form express is available only when background noise is ignored.

Still, we first study the transmission success probability $p_s(d)$ for a given uplink between a BS at the origin and a device $x_0$ at distance $d$. Using Slivnyak's theorem~\cite{vaze2015random}[Theorem 2.3.3], combining $(\ref{eq:path-loss})$ and $(\ref{eq: sinr-definition})$, we have:
\begin{align}
p_{s} \left( d \right)
& =\mathbb{P}\left\lbrace \frac{H \exp(\chi) d^{-\gamma}}{\sum_{x_j \in \Phi_{m}} H_{x_j} \exp(\chi_{x_j}) d_{x_j}^{-\gamma} + N}  \geq \theta_T \right\rbrace. \nonumber
\end{align}
Let $I=\sum_{x_j \in \Phi_{m}} H_{x_j} \exp(\chi_{x_j}) d_{x_j}^{-\gamma}$, which is the cumulative interference suffered by $x_0$. Conditioned on log-normal shadowing component $\exp(\chi_{x_j}) $, as shown in~\cite{haenggi2009interference}, $p_{s}\left( d \right)$ can be expressed in terms of Laplace transform of cumulative interference $\mathcal{L}_{I}(s)$ at point $\theta_{T} e^{-\chi} d^{\gamma}$:
\begin{align}
\label{eq:def_ps_nearest_bs}
p_{s}\left( r \right)  &= Pr \left\lbrace H  \geq (I+N) \theta_{T} e^{-\chi}d ^{\gamma}  \right\rbrace \nonumber\\ 
% The follwing line can be hidden, not necessary in FINAL_VERSION
%&=\mathbb{E}_{\chi}\left\lbrace  \mathbb{E}_{I} \left[ \exp(-\theta_{T} \exp(-g) r^{\gamma}  I ) \vert \chi = g\right]\right\rbrace   \nonumber\\
&= \int_{-\infty}^{+\infty}   \exp(-N \theta_{T} e^{-x} r^{\gamma} ) \left[ \int_{0}^{+\infty} \exp(-y \theta_{T} e^{-x} r^{\gamma} ) f_{I}(y)dy\right] f_{\chi}(x) dx  \nonumber\\
&= \mathbb{E}_{\chi} \left[ \exp(-N \theta_{T} e^{-\chi} d^{\gamma} ) \mathcal{L}_{I}\left\lbrace \theta_{T} e^{-\chi} d^{\gamma}\right\rbrace \right],
\end{align}
where $f_{X} (x)$ is probability density function (PDF) of r.v. $X$, $\mathbb{E}_{X} \left[ \cdot \right] $ is the expectation operator with respect to $X$.
%This method is rational when downlink is seldom used for the sake of cost. %TODO: Add this in cover letter.

Therefore, network packet loss rate $P_{f,n}$ (index $n$ means nearest) is the expectation of $p_s(d)$ with respect to the distance to the nearest base station $r$ . The PDF of $d$ for a PPP, proved in~\cite{andrews2011tractable}, is as follows:
\begin{align}
\label{eq:pdf_nearest_distance}
f\left( d \right)  = 2 \pi \lambda_b  d \exp(-\lambda_b \pi d^2), d\in \left[ 0, +\infty\right]. 
\end{align}
Thus:
\begin{align}
\label{eq:mean_bx+1_step1}
&P_{f,n}= \mathbb{E}_{r}\left[ 1-p_{s}\left(d\right) \right]  \nonumber\\
&= 1 -\int_{0}^{+\infty} \mathbb{E}_{\chi} \left[ \exp(-p \lambda_{m} \pi A \theta_{T}^{\frac{2}{\gamma}} e^{\frac{2\sigma^2}{\gamma^2}}  d^2 e^{-\frac{2}{\gamma}\chi}) \right] 2 \pi \lambda_b  d e^{-\lambda_b \pi d^2} d(d) \nonumber\\
&= 1-\pi \lambda_b \mathbb{E}_{\chi}\left[ \int_{0}^{+\infty} \exp(-p \lambda_{m} \pi A \theta_{T}^{\frac{2}{\gamma}} e^{\frac{2\sigma^2}{\gamma^2}}  d^2 e^{-\frac{2}{\gamma}\chi}-\lambda_b \pi d^2)\right] d(d^2) \nonumber\\
&= 1 -\mathbb{E}_{\chi}\left[\left( A \theta_{T}^{\frac{2}{\gamma}} e^{\frac{2\sigma^2}{\gamma^2}} L  e^{-\frac{2}{\gamma}\chi}+1 \right)^{-1} \right] \nonumber\\
%&= 1- \int_{-\infty}^{+\infty} \frac{1}{ \frac{p\lambda_{m}AK}{\pi \lambda_{b}}e^t+1}\frac{\exp\left\lbrace -\frac{t^2}{2 \left( \frac{2}{\gamma}\sigma\right) ^2}\right\rbrace }{\sqrt{2\pi} \frac{2}{\gamma}\sigma} dt \nonumber\\
&= 1 - \int_{-\infty}^{+\infty} \!\!\! \frac{\gamma}{ 2\sqrt{2\pi}\left( e^t+1\right)\sigma} \exp \left\lbrace -\frac{\gamma^2}{8 \sigma^2} \left(    t-\ln(A L \theta_{T}^{\frac{2}{\gamma}} e^{\!\! \frac{2\sigma^2}{\gamma^2}} )   \right) ^2 \right\rbrace dt.  
\end{align}
%which is actually a logistic-normal integral. According to~\cite{crooks2009logistic}, this kind of 
Integral in $(\ref{eq:mean_bx+1_step1})$ can be accurately approximated by a logistic function~\cite{crooks2009logistic}. Thus,
\begin{align}
\label{eq:bs_nst_att_analytical}
P_{f,n} 
&\approx 1 - \frac{1}{1 + \exp\left\lbrace \left( 1 +\pi \sigma^2 / (2\gamma^2) \right)^{-1/2} \ln( A \theta_{T}^{\frac{2}{\gamma}} e^{\frac{2\sigma^2}{\gamma^2}}  L) \right\rbrace}  \nonumber\\
&\approx 1-\frac{ 1 }{ 1 + \left( A \theta_{T}^{\frac{2}{\gamma}} e^{\frac{2\sigma^2}{\gamma^2}}  L\right) ^{C} }
\end{align}
where $A$ is defined in $(\ref{eq:def_ps_2})$, $L=p\lambda_{m}/\lambda_{b}$, $C= \left( 1 +\pi \sigma^2 / (2\gamma^2) \right)^{-1/2}$. We conducted a Monte-Carlo simulation and found that the maximum difference between $(\ref{eq:mean_bx+1_step1})$ and $(\ref{eq:bs_nst_att_analytical})$ is $2.46\%$ in normalized load interval $\left[ 0.021, 0.3\right] $, which proves the accuracy of the proposed approximation formula $(\ref{eq:bs_nst_att_analytical})$
%\qs{Remember to put the figure into Annexe part. Here I wan to show that, the difference between approximation formula and exact integral is smaller when normalized load increase }.


From $(\ref{eq:bs_nst_att_analytical})$,  the maximum supported normalized load $L_{\text{max},n}$ and normalized spatial throughput (index $n$ refers to the nearest BS attach method) is obtained:
\begin{align}
\label{eq:bs_nst_att_max_load}
L_{\text{max},n} & \approx \frac{1}{A \theta_{T}^{\frac{2}{\gamma}} e^{\frac{2\sigma^2}{\gamma^2}} } 
\left(  \frac{P_{f}^{\text{max}}}{1 - P_{f}^{\textbf{max}}} \right) ^{\frac{1}{C}}, \\
\label{eq:bs_nst_att_spatial_throughput}
S_{n} & \approx  \frac{L}{1 + \left( A \theta_{T}^{\frac{2}{\gamma}} e^{\frac{2\sigma^2}{\gamma^2}}  L\right) ^{C} }.
\end{align}
It should be noted that according to approximation error, $\eqref{eq:bs_nst_att_spatial_throughput}$ is not valid when normalized load $L$ is very high. A concise explication is as follows: let $\hat{P_{f,n}}$ and $o(L)$ respectively be the real packet loss rate value and approximation error, thus $\hat{P_{f,n}} = P_{f,n} + o(L)$. Thus, the approximation error between real spatial throughput $\hat{S_n}$ and $\eqref{eq:bs_nst_att_spatial_throughput}$ is thus $ o(L) L$, which is not neglected as the increase of $L$.
%We should note that $(\ref{eq:bs_nst_att_spatial_throughput})$ is obtained by $S = (1-P_{f}) p\lambda_{m}/ \lambda_{b}$, where $P_{f}$ is a ... This comment is applied to all spatial throughput in BS attach case.

\subsection{Best BS attach method}
In this subsection, we assume that the device attaches to the BS for which the received power averaged over all fading realizations is the strongest (i.e. the BS that maximizes $r^{-\gamma}\exp(\chi)$). The displacement theorem, discussed in Sec.~\ref{subsec:displacement_theorem} can be applied. Let $P_{f,b} $ be packet loss rate for this case. In a PPP formed by BS's location with intensity $\lambda_b e^{\frac{2\sigma^2}{\gamma^2}}$, the distance $r$ between a device placed at the origin and its nearest BS is as follows:
\begin{align}
	\label{eq:pdf_modified_r}
	f(r) = 2 \pi \lambda_b e^{\frac{2\sigma^2}{\gamma^2}}  \exp( -\lambda_b  e^{\frac{2\sigma^2}{\gamma^2}} \pi r^2 ) r 
\end{align}
\subsubsection{With background noise and $\gamma$=4}
\label{subsub:with_background_noise}
When background noise is taken into account, only and only if path loss component $\gamma=4$, we obtain the closed-form expression. 
\begin{align}
	\label{eq:bs_best_att_analytical_with_noise_step_1}
	&P_{f,b}= \mathbb{E}_{r}\left[ 1-p_{s}\left(r\right) \right]  \nonumber\\
	&= 1 -\int_{0}^{+\infty}   \exp(-N \theta_{T} r^{4} ) \exp(-p \lambda_{m} \pi A \theta_{T}^{\frac{2}{\gamma}} e^{\frac{2\sigma^2}{\gamma^2}}  r^2 ) 2 \pi \lambda_b e^{\frac{2\sigma^2}{\gamma^2}}  \exp( -\lambda_b  e^{\frac{2\sigma^2}{\gamma^2}} \pi r^2 ) r dr \nonumber\\
	&= 1 -\pi \lambda_b e^{\frac{2\sigma^2}{\gamma^2}} \int_{0}^{+\infty}   \exp(-N \theta_{T} r^{4} -(p \lambda_{m} \pi A \theta_{T}^{\frac{2}{\gamma}} e^{\frac{2\sigma^2}{\gamma^2}}  + \lambda_b \pi  e^{\frac{2\sigma^2}{\gamma^2}} )r^2 )  dr^2
\end{align}
Integral $\eqref{eq:bs_best_att_analytical_with_noise_step_1}$ is finally simplified as follows. The detailed steps are in Annexe~\ref{annexe:integration-with-background-noise}:
\begin{align}
	\label{eq:bs_best_att_analytical_with_noise}
	P_{f,b}
	&= 1 - \pi \lambda_b e^{\frac{2\sigma^2}{\gamma^2}} \frac{\sqrt{\pi}}{2} \sqrt{\frac{1}{U}}\exp(\frac{V^2}{4U})\left[ 1 - \erf(\frac{V}{2\sqrt{U}})\right],
\end{align}
where $U=N \theta_{T}$, $V=\exp \left( \frac{2\sigma^2}{\gamma^2}  \right) \left( p \lambda_{m} \pi A \theta_{T}^{\frac{2}{\gamma}} + \pi \lambda_b \right)$

\subsubsection{Without Background Noise}
With ignorance of background noise, the link level transmission success probability $p_{s}(r)$, given in~\eqref{eq:def_ps_2}, is simplified as follows:
\begin{align}
	\label{eq:succ_proba_with_modified_r}
	p_{s}(r) =\exp(-p \lambda_{m} \pi A e^{\frac{2\sigma^2}{\gamma^2}} \theta_{T}^{\frac{2}{\gamma}} r^2 ).
\end{align}
Similar to Sec~\ref{subsub:with_background_noise}, we have:
\begin{align}
\label{eq:bs_best_att_analytical}
&P_{f,b}= \mathbb{E}_{r}\left[ 1-p_{s}\left(r\right) \right]  \nonumber\\
&= 1 -\int_{0}^{+\infty}  \exp(-p \lambda_{m} \pi A \theta_{T}^{\frac{2}{\gamma}} e^{\frac{2\sigma^2}{\gamma^2}}  r^2 )  2 \pi \lambda_b e^{\frac{2\sigma^2}{\gamma^2}}  \exp( -\lambda_b  e^{\frac{2\sigma^2}{\gamma^2}} \pi r^2 ) r dr \nonumber\\
&= 1-\frac{1}{ 1 +  A \theta_{T}^{\frac{2}{\gamma}} L }.
\end{align}
By inverting $(\ref{eq:bs_best_att_analytical})$, the maximum supported normalized load $L_{\text{max}, b}$ (subscript $b$ refers to the best BS attach method) is as follows:
\begin{align}
	\label{eq:best_maximum_load}
	L_{\text{max},b} &=\frac{1}{A \theta_{T}^{\frac{2}{\gamma}}  } 
	\frac{P_{f}^{\text{max}}}{1 - P_{f}^{\textbf{max}}}. 
\end{align}
The normalized spatial throughput $S_{b}$ is as follows:
\begin{align}
	S_{b} =  \frac{L}{1 +  A  \theta_{T}^{\frac{2}{\gamma}} L }.
\end{align}



\subsection{Selective Combining like Macro Diversity}
\label{sec:sc_macro_diversity}
\subsubsection{With Background Noise and $\gamma=4$}
Since a transmitted packet is received by all surrounding BS, the transmission of this packet fails if and only if the received SINR at each BS is less than the capture ratio. In other words, if the maximum SINR is still less than the capture ratio, the packet transmission is failed. This is why such a scheme is called selective combining (SC) like macro reception diversity. In this section, we evaluate SC-like macro diversity in one-shot random access case.

Strictly speaking, the cumulative interferences at two different BS are correlated: when a device is transmitting, it generates some non-negligible interference on base stations that are not very far. However, it is still rational to assume that the interferences received by different BS are mutually independent.
%when received power at each BS exhibits obvious diversity such as in urban area suffered by strong shadowing effect. 
The reasons are in twofold:\begin{inparaenum}[1)]
	\item the interference correlation coefficient is shown to be close to $0$ if locations of two BS are different with path-loss model $r^{-\gamma}$~\cite[lemma 3.5]{haenggi2009interference}; 
	\item Each contribution to the cumulative interference is affected by fading and shadowing, which are i.i.d. random variable for different base stations.
\end{inparaenum}
Thus, the packet loss rate $P_{f,m}$ (index $m$ refers to macro reception diversity) is the expectation of the product of failure probabilities to each BS:
\begin{align}
\label{eq:definition_pfm}
P_{f,m} &= \mathbb{E}\left[  \prod_{r_i \in \Phi_{b}} (1-p_{s}(r_i)) \right], \text{with } r_i \in \left[0, +\infty\right],
\end{align} 
where $r_i$ is the distance between the device and BS with label $i$. Using Campbell theorem for  $(\ref{eq:definition_pfm})$:
\begin{align}
	\label{eq:bs_rx_divers_analytical_with_noise_before_last}
	P_{f,m} &= \exp\left\lbrace -2\pi \lambda_{b} e^{\frac{2\sigma^2}{\gamma^2}}  \int_{0}^{+\infty} p_{s}(r)rdr \right\rbrace \nonumber\\
	&= \exp \left\lbrace  -2\pi \lambda_{b} e^{\frac{2\sigma^2}{\gamma^2}}   \int_{0}^{+\infty}   \exp(-N \theta_{T} r^{4} ) \exp(-p \lambda_{m} \pi A \theta_{T}^{\frac{2}{\gamma}} e^{\frac{2\sigma^2}{\gamma^2}}  r^2 )    r dr   \right\rbrace \nonumber \\
	&= \exp \left\lbrace  -\pi \lambda_{b} e^{\frac{2\sigma^2}{\gamma^2}}  \int_{0}^{+\infty}   \exp(-N \theta_{T} r^{2} -p \lambda_{m} \pi A \theta_{T}^{\frac{2}{\gamma}} e^{\frac{2\sigma^2}{\gamma^2}}  r ) dr   \right\rbrace
\end{align} 
With mathematical operations shown in Appendix $\eqref{annexe:integration-with-background-noise}$, we finally arrive at:
\begin{align}
	\label{eq:bs_rx_divers_analytical_with_noise}
	P_{f,m}  &= \exp\left\lbrace -\pi \lambda_{b} e^{\frac{2\sigma^2}{\gamma^2}} \frac{\sqrt{\pi}}{2} \sqrt{\frac{1}{U}}\exp(\frac{V^2}{4U})\left[ 1 - \erf(\frac{V}{2\sqrt{U}})\right]  \right\rbrace    ,
\end{align} 
where $U=N \theta_{T}, V=p \lambda_{m} \pi A \theta_{T}^{\frac{2}{\gamma}} \exp( \frac{2\sigma^2}{\gamma^2})$

\subsubsection{Without Background Noise}
In case of background noise is neglected (i.e., $N=0$), combining with $\eqref{eq:succ_proba_with_modified_r}$ and changing integration order, $\eqref{eq:bs_rx_divers_analytical_with_noise_before_last}$ can be directly simplified:
\begin{align}
	\label{eq:bs_rx_divers_analytical}
	P_{f,m} 
	%&= \exp\left\lbrace -\left[A \theta_{T}^{\frac{2}{\gamma}} L \right] ^{-1}\right\rbrace.
	&= \exp\left\lbrace -\frac{1}{A \theta_{T}^{\frac{2}{\gamma}} L }\right\rbrace.
\end{align}
From $\eqref{eq:bs_rx_divers_analytical}$, we can easily find the maximum load and throughput:
\begin{align}
	\label{eq:diversity_maximum_load}
	L_{\text{max}, m} = \frac{1}{A \theta_{T}^{2/\gamma}} \frac{1}{\ln(P_{f}^{\textbf{max}})}
\end{align}
\begin{align}
	S_{m} = L \left( 1 - \exp\left\lbrace -\frac{1}{A \theta_{T}^{\frac{2}{\gamma}} L }\right\rbrace \right) 
\end{align}
The macro diversity gain against the best BS attach mode $G_{\text{diversity}, b}$ is obtained by the ratio $\eqref{eq:diversity_maximum_load}$ and $\eqref{eq:best_maximum_load}$:
\begin{align}
	G_{\text{diversity}, b} = (1 - P_{f}^{\text{max}}) / \left( P_{f}^{\text{max}} \ln(1/P_{f}^{\text{max}})\right). 
\end{align}

\subsection{Maximum Ratio Combining based Macro diversity}
Due to the fact that a transmitted packet theoretically can be simultaneously received by all BS (ignorance of background noise), linear combining of signals at each BS can be leveraged so that the output SINR is maximized. Such a scheme is called Maximum Ratio Combining (MRC) based macro diversity
In this section, the performance of MRC-based macro diversity in case of one-shot random access is evaluated. The background noise is neglected for the sake of tractability.

Consider a typical device $x_0$ at origin, it has been proved that if the weigh factors involved in MRC context is well designed (see Fig.~\ref{fig:mrc_macro_diversity_recpetion_illustration}), the output $\text{SINR}$ $\Theta$ of best combiner is expressed as 
%TODO: need a reference here...
%\qs{We need a reference for this...}:
\begin{align}
\Theta &= \sum_{y_j \in \Phi_{b}}^{} \theta_{y_j} \nonumber\\
&= \sum_{y_j \in \Phi_{b}}^{} \frac{H_{y_j} \exp(\chi_{y_j}) r_{y_j}^{-\gamma}}{I_{y_j}} \nonumber\\
&= \sum_{y_j \in \Phi_{b}}^{} \epsilon_{y_j} r_{y_j}^{-\gamma}
\end{align}
where $\theta_{y_j}$ is the received SINR at BS $y_j$, $H_{y_{j}}$ and $\exp(\chi_{y_j})$ are respectively the Rayleigh fading and shadowing component for link between device $x_0$ and BS $y_j$, $I_{y_j}$ refers to cumulative interference suffered at BS $y_j$. For the sake of notation simplicity, let $\epsilon_{y_j}  = H_{y_j}\exp(\chi_{y_j})  / I_{y_j}$. Using similar justification explained in Sec.~\ref{sec:sc_macro_diversity}, $\epsilon_{y_j}$ for $j=0,1,2,...$ constitute a series of RV whose element is assumed to be identically independently distributed. 

Let $P_{f, \text{MRC}}$ be the network level packet loss rate in this case. According to capture effect:
\begin{align}
	P_{f, \text{MRC}} &= \mathbb{P} \left\lbrace \Theta < \theta_{T} \right\rbrace \nonumber \\
	&= F_{\Theta} (\theta_{T}),
\end{align}
where $\theta_{T}$ is the capture ratio, $F_{\Theta} (\theta)$ is the CDF of $\Theta$. It is can be numerically computed from Laplace Transform or Characteristic function of $\Theta$.

The Laplace Transform (LT) of $\Theta$ is by definition as follows
%TODO:....
%x\qs{To be more restrict, we should explain why we can remove the subscript $y_i$ in the following formula}:
\begin{align}
\label{eq:lt-sinr-mrc-1}
\mathcal{L}_{\Theta}\left( s \right) &= \mathbb{E}\left[ e^{-s\Theta}\right] \nonumber\\
&=\mathbb{E}\left[ \exp( -s \sum_{y_j \in \Phi_{b}}^{} \theta_{y_j} )\right] \nonumber\\
&= \mathbb{E}\left[ \prod_{y_j \in \Phi_b}^{} \mathbb{E}_{\epsilon} \left[ \exp( -s \epsilon r_{y_{j}}^{-\gamma}) \right] \right] 
\end{align}
Actually, the mathematical operations for interference analysis used in~\cite{haenggi2009interference} can be reused for $\eqref{eq:lt-sinr-mrc-1}$. For the sake of clarity, we detail the operations as follows:
 
Applying Campbell theorem to $\eqref{eq:lt-sinr-mrc-1}$ and changing the order of integration and expectation operator, we have:
\begin{align}
\label{eq:lt-sinr-mrc-2}
\mathcal{L}_{\Theta}\left( s \right) &= \exp \left\lbrace -\int_{0}^{+\infty}    \mathbb{E}_{\epsilon}\left[ 1-\exp(-s\epsilon r^{-\gamma} ) \right]   2 \pi \lambda_{b} r dr \right\rbrace \nonumber\\
&= \exp \left\lbrace -\mathbb{E}_{\epsilon}\left[ \int_{0}^{+\infty} \left(1-\exp(-s\epsilon r^{-\gamma} ) \right) 2 \pi \lambda_{b} r dr  \right]   \right\rbrace 
\end{align}

Let us focus on the integral $D = \int_{0}^{+\infty} \left(1-\exp(-s\epsilon r^{-\gamma} ) \right) 2 \pi \lambda_{b} r dr $:
\begin{align}
\label{itg:D}
D
&\overset{\mathclap{\strut\text{(a)}}} = \pi \lambda_{b} \int_{0}^{+\infty} \left(1-\exp(-x ) \right) d\left( -x^{-\frac{2}{\gamma}} (s\epsilon)^{\frac{2}{\gamma}}\right)  \nonumber\\
&\overset{\mathclap{\strut\text{(b)}}}= \pi \lambda_{b} \int_{0}^{+\infty} \exp(-x ) x^{-\frac{2}{\gamma}} (s\epsilon)^{\frac{2}{\gamma}} dx \nonumber\\
&= \pi \lambda_{b}  (s\epsilon)^{\frac{2}{\gamma}}  \Gamma (1 - \frac{2}{\gamma}),
\end{align}
where step $(a)$ is obtained with a change of variable $x = s \epsilon r^{-\gamma}$, step $(b)$ is achieved via integration by parts, $\Gamma(\cdot)$ is gamma function.

Combining $\eqref{eq:lt-sinr-mrc-2}$ and $\eqref{itg:D}$ , the Laplace Transform of $\Theta$ is finally simplified as:
\begin{align}
\label{eq:lt_Theta}
\mathcal{L}_{\Theta}\left( s \right) &= \exp(-\lambda_{b} \pi \mathbb{E}\left[ \epsilon ^{\frac{2}{\gamma}} \right]  \Gamma(1-\frac{2}{\gamma}) s^{\frac{2}{\gamma}}),
\end{align}
where $\mathbb{E}\left[ \epsilon ^{\frac{2}{\gamma}} \right] $ is:
\begin{align}
\label{eq:epsilon_fractional_moment}
\mathbb{E}\left[ \epsilon ^{\frac{2}{\gamma}} \right]  &= \mathbb{E}\left[ \left( H \exp(\chi)\right)  ^{\frac{2}{\gamma}} \right] \mathbb{E}\left[ I ^{-\frac{2}{\gamma}}\right] \nonumber\\
&=\Gamma(1+\frac{2}{\gamma}) \exp( \frac{2\sigma^2}{\gamma^2}) \mathbb{E}\left[ I ^{-\frac{2}{\gamma}}\right] 
\end{align}

Since the Laplace transform of cumulative interference $I$ can be easily obtained and PDF of $I$ is not always exist except $\gamma=4$, to compute $\mathbb{E}\left[ I ^{-\frac{2}{\gamma}}\right]$ (i.e. negative fractional moment calculation problem), one can consider departure from LT. This is a research subject in applied mathematical domain and planned as our future work. The most straightforward way is to rely on Monte-Carlo method.

With substitution $s = -i \omega$ in $\eqref{eq:lt_Theta}$ , the characteristic function (CF) of $\Theta$ is as follows:
\begin{align}
\phi_{\Theta}\left( \omega \right) &= \exp(-\lambda_{b} \pi \mathbb{E}\left[ \epsilon ^{\frac{2}{\gamma}} \right]  \Gamma(1-\frac{2}{\gamma}) \exp(-i\pi/\gamma) \omega^{\frac{2}{\gamma}}),  
\end{align}
where $\omega \geq 0$, $i$ is imaginary unit.

%TODO:
%\qs{This paragraph should be simplified. Since in another Chapter, the same numerical technique should have been presented...}
As a continuous random variable, the cumulative distribution function $F_{\Theta}\left( \theta \right)$ of total SINR $\Theta$ can be directly derived from its characteristic function $\phi_{\Theta}\left(\omega\right)$. Applying mathematical techniques used in finance domain~\cite{hirsa2012computational}, we seek to calculate the Fourier transform of $e^{-\eta \theta} F_{\Theta}\left( x \right)$ where term $e^{-\eta \theta}$ is a damping function with $\eta > 0$. 
\begin{align}
\label{eq:intermediate_formula_1}
\int_{-\infty}^{+\infty} e^{i\omega \theta} e^{-\eta \theta} F_{\Theta}\left( \theta \right) dx = \frac{1}{\eta - i\omega} \phi_{\Theta}\left( \omega +i\eta \right) 
\end{align}
Applying Fourier inversion for $\eqref{eq:intermediate_formula_1}$, we obtain the expression for $F_{\Theta}\left( \theta \right)$ as follows:
\begin{align}
\label{eq:pr_c_m_case2}
F_{\theta}\left( \theta \right)  &= \frac{e^{\eta \theta}}{2\pi} \int_{-\infty}^{+\infty} e^{-i \omega \theta} \frac{1}{\eta - i\omega} \phi_{Theta}\left( \omega +i\eta\right) d\omega  \nonumber\\
&= \frac{e^{\eta \theta}}{\pi} \Re\left\lbrace  \int_{0}^{+\infty} e^{-i \omega \theta} \frac{1}{\eta - i\omega} \phi_{\Theta}\left( \omega +i\eta\right) d\omega\right\rbrace, 
\end{align}
The packet loss rate $P_{f,\text{MRC}}$ in case of capture ratio $\theta_{T}$ can be obtained by substituting $\theta = \theta_{T}$ in $\eqref{eq:pr_c_m_case2}$. The latter can be computed using a single numerical integration.


%We need to calculate CDF of $\Theta$ which is the packet loss rate. However, unlike in Sec.XXX, in general case, no closed-form expression. We rely on numerical method to compute packet loss rate wit low complexity. blabla...

\subsubsection{A special case, $\gamma=4$, without background noise}
In this section, we study a special case where path loss component $\gamma=4$. Proved in ~\cite[Eq. 3.17]{haenggi2009interference}, the probability density function of cumulative interference $I$, suffered by device at the origin, in a system without shadowing, is as follows:
\begin{align}
f_{I}(x) = \frac{p\lambda_{m}}{4} (\frac{\pi}{x})^{\frac{3}{2}} \exp(-\frac{\pi^4 p^2\lambda_{m}^2}{16x}), x \geq 0
\end{align}
Along side with log-normal shadowing effect, the probability density function of cumulative interference $I$ is obtained just by scaling $\lambda_{m}$ as $ \lambda_{m} \exp(\frac{\sigma^2}{8})$:
\begin{align}
f_{I}(x) = \frac{    p\lambda_{m}   \exp(\frac{\sigma^2}{8})  }{4} (\frac{\pi}{x})^{\frac{3}{2}} \exp(    -\frac{    \pi^4    p^2\lambda_{m}^2     \exp^2(\frac{\sigma^2}{8})    }{    16x    }    ), x \geq 0
\end{align} 
Hence, the negative fractional moment of $I$ can be calculated:
\begin{align}
\label{eq:gamma=4_fractional_I}
\mathbb{E}\left[ I ^{-\frac{1}{2}}\right] &= \int_{0}^{+\infty} x^{-\frac{1}{2}} f_{I}(x) dx \nonumber\\
&= \int_{0}^{+\infty}  \frac{    p\lambda_{m}   \exp(\frac{\sigma^2}{8})  }{4}   \pi^{\frac{3}{2}}  x^{-2}  \exp(    -\frac{    \pi^4    p^2\lambda_{m}^2     \exp^2(\frac{\sigma^2}{8})    }{    16x    }    )  dx \nonumber\\
&= \frac{4}{    \pi^{\frac{5}{2}}  p\lambda_{m} \exp(\frac{\sigma^2}{8}) }
\end{align}
With substitution of $\eqref{eq:gamma=4_fractional_I}$ into $\eqref{eq:epsilon_fractional_moment}$,
\begin{align}
\label{eq:negative_fractonal_epsilon_4}
\mathbb{E}\left[ \epsilon ^{\frac{1}{2}} \right] & = 4\Gamma(\frac{3}{2})\pi^{-\frac{5}{2}} p^{-1}\lambda_{m}^{-1}\nonumber\\
&=2\pi^{-2} p^{-1}\lambda_{m}^{-1}
\end{align}
From $\eqref{eq:negative_fractonal_epsilon_4}$ and $\eqref{eq:lt_Theta}$, we have:
\begin{align}
\mathcal{L}_{\Theta}\left( s \right) &= \exp(-\lambda_{b} \pi 2\pi^{-2}\lambda_{m}^{-1} \Gamma(\frac{1}{2}) s^{\frac{1}{2}}) \nonumber \\
&= \exp(-2L^{-1}\pi^{-\frac{1}{2}}  s^{\frac{1}{2}}), 
\end{align}
which is the Laplace Transform of probability density function of $\Theta$. According to derivative property of Laplace Transform, the cumulative distribution function $F_{\Theta} (\theta) $ of ${\Theta}$ can be obtained by directly inversing $\frac{1}{s} \mathcal{L}_{\Theta}\left( s \right)$.
When $\gamma = 4$, instead of using numerical integration for $\eqref{eq:pr_c_m_case2}$, we have closed-form CDF of total SINR $\Theta$:
\begin{align}
P_{f, \text{MRC}}  = F_{\Theta} (\theta_{T}) 
&= \mathcal{L}^{-1} \left[ \frac{1}{s} \mathcal{L}_{\Theta}\left( s \right) \right]  \nonumber\\
&=1 -\erf \left( 
\frac{1}{ L \sqrt{\pi \theta_{T}}}
\right),
\end{align}
where $\mathcal{L} \left[ \cdot \right] $ is inverse Laplace Transform operator.
%We verify that the packet loss rate obtained by $(?)$ and $(?)$ are the same.
\qs{Don't forget to talk about throughput, macro diversity gain as like what we have done in previous section.}

%============================================================================================
%\subsection{Maximum Ratio Combining based Macro diversity}
Due to the fact that a transmitted packet theoretically can be simultaneously received by all BS (ignorance of background noise), linear combining of signals at each BS can be leveraged so that the output SINR is maximized. Such a scheme is called Maximum Ratio Combining (MRC) based Macro diversity\qsong{It is better to give a reference to say this is studied by some LPWAN networks}. In this section, the performance of MRC-based Macro diversity in one-shot random access is evaluated.

In the literature, it has been prove that if the weigh factors involved in MRC context is well designed, the output $\text{SINR}$ $\Theta$ of best combiner is expressed as \qsong{Perhaps we also need a reference for this...}:
\begin{align}
	\Theta = \sum_{y_j \in \Phi_{b}}^{} \text{SINR}_{y_j},
\end{align}
where $\text{SINR}_{y_j}$ is the received SINR at BS with label $y_j$.

The Laplace Transform (LT) of $\Theta$ is by definition as follows:
\begin{align}
	\label{eq:lt-sinr-mrc-1}
	\mathcal{L}_{\Theta}\left( s \right) &= \mathbb{E}\left[ e^{-s\Theta}\right] \nonumber\\
	&=\mathbb{E}\left[ \exp( -s \sum_{y_j \in \Phi_{b}}^{} \Theta_{y_j} )\right]
\end{align}

Recall that the received SINR of target BS is defined as:
\begin{align}
	\theta &= \frac{H \exp(\chi) r^{-\gamma}}{I} \nonumber\\
	&= \epsilon r^{-\gamma}
\end{align}
where term $\epsilon = H\exp(\chi)  / I$ is identically independently distributed RV for each BS.
\begin{align}
	\label{eq:lt-sinr-mrc-2}
	\mathcal{L}_{\Theta}\left( s \right) &= \mathbb{E}\left[ \prod_{y_j \in \Phi_b}^{} \mathbb{E}_{\epsilon} \left[ \exp( -s \epsilon r^{-\gamma}) \right] \right] 
\end{align}
Applying Campbell theorem to $(\ref{eq:lt-sinr-mrc-1})$,
\begin{align}
	\mathcal{L}_{\Theta}\left( s \right) &= \exp \left\lbrace -\mathbb{E}_{\epsilon}\left[ \int_{0}^{+\infty} \left(1-\exp(-s\epsilon r^{-\gamma} )2 \pi \lambda_{b} r dr\right)  \right]   \right\rbrace 
\end{align}

Let us focus on the integral $D = \int_{0}^{+\infty} \left(1-\exp(-s\epsilon r^{-\gamma} ) \right) 2 \pi \lambda_{b} r dr $:
\begin{align}
	\label{itg:D}
	D &= \int_{0}^{+\infty} \left(1-\exp(-s\epsilon r^{-\gamma/2} ) \right) \pi \lambda_{b} dr \nonumber\\
	&= \pi \lambda_{b} \int_{0}^{+\infty} \left(1-\exp(-s\epsilon r^{-\gamma/2} ) \right) \frac{2}{\gamma} x^{\frac{2}{\gamma}  - 1} dx \text{ (substitution $x = r^{\gamma/2}$)}
\end{align}
To further calculate $(\ref{itg:D})$, we note that it is the expected value of RV $ \left(    (X/\epsilon s) ^{-1} \right)^{\frac{2}{\gamma}} $ where $X$  is an exponential RV with unit mean. Since $\mathbf{E}\left[ X^{p}\right] = \Gamma(1+p)$ by the definition of the gamma function. Hence, we have:
\begin{align}
	\mathbf{E} \left[ \left(    (X/\epsilon s) ^{-1} \right)^{\frac{2}{\gamma}}  \right] = (s \epsilon) ^{\frac{2}{\gamma}} \Gamma (1 - \frac{2}{\gamma}) \nonumber.
\end{align}
 
Finally, $(\ref{eq:lt-sinr-mrc-2})$ is simplified as:
\begin{align}
	\mathcal{L}_{\Theta}\left( s \right) &= \exp(-\lambda_{b} \pi \mathbb{E}\left[ \epsilon ^{\frac{2}{\gamma}} \right]  \Gamma(1-\frac{2}{\gamma}) s^{\frac{2}{\gamma}})
\end{align}

Let us now focus on term $\mathbb{E}\left[ \epsilon ^{\frac{2}{\gamma}} \right] $:
\begin{align}
	\mathbb{E}\left[ \epsilon ^{\frac{2}{\gamma}} \right]  = \mathbb{E}\left[ \left( H \exp(\chi)\right)  ^{\frac{2}{\gamma}} \right] \mathbb{E}\left[ I ^{-\frac{2}{\gamma}}\right] 
\end{align}

For term $\mathbb{E}\left[ I ^{-\frac{2}{\gamma}}\right] $, it is a negative fractional moment calculation problem, which is also a research subject in applied mathematical domain. \qsong{Still need to find some numerical technique in the literature to compute this. I think it is the last difficulty to be solved if we want to obtain a general analytical framework...}

With substitution $s = -j\omega$, the characteristic function (CF) of $\Theta$:
\begin{align}
\phi_{\Theta}\left( \omega \right) &= \exp(-\lambda_{b} \pi \mathbb{E}\left[ \epsilon ^{\frac{2}{\gamma}} \right]  \Gamma(1-\frac{2}{\gamma}) \exp(-i\pi/\gamma) \omega^{\frac{2}{\gamma}}),  
\end{align}
where $\omega \geq 0$.

As a continuous random variable, the cumulative distribution function $F_{\Theta}\left( x \right)$ of total SINR $\Theta$ can be directly derived from its characteristic function $\phi_{\Theta}\left(\omega\right)$, for example by use of Gil-Pelaez Theorem~\cite{gil1951note}. However, directly using Gil-Pelaez Theorem needs long time. Applying mathematical techniques used in finance domain~\cite{hirsa2012computational}, we seek to calculate the Fourier transform of $e^{-\eta x} F_{\Theta}\left( x \right)$ where term $e^{-\eta x}$ is a damping function with $\eta > 0$. 
\begin{align}
\label{eq:intermediate_formula_1}
\int_{-\infty}^{+\infty} e^{i\omega x} e^{-\eta x} F_{Y_k}\left( x \right) dx = \frac{1}{\eta - iw} \phi_{Y_{k}}\left( \omega +i\eta \right) 
\end{align}
Applying Fourier inversion for ($\ref{eq:intermediate_formula_1}$), we obtain the expression for $F_{\Theta}\left( x \right)$ as follows:
\begin{align}
\label{eq:pr_c_m_case2}
F_{\theta}\left( x \right)  &= \frac{e^{\eta x}}{2\pi} \int_{-\infty}^{+\infty} e^{-i \omega x} \frac{1}{\eta - i\omega} \phi_{\theta}\left( \omega +i\eta\right) d\omega  \nonumber\\
&= \frac{e^{\eta x}}{\pi} \Re\left\lbrace  \int_{0}^{+\infty} e^{-i \omega x} \frac{1}{\eta - i\omega} \phi_{\theta}\left( \omega +i\eta\right) d\omega\right\rbrace, 
\end{align}
The cumulative distribution function $F_{\theta}\left( x \right)$ now can be derived directly from ($\ref{eq:pr_c_m_case2}$) using a single numerical integration.

\subsection{A special case, $\gamma=4$}
In this case, we study a special case where $\gamma=4$.
\begin{align}
	\mathbb{E}\left[ \left( H \exp(\chi)\right)  ^{\frac{1}{2}} \right]  = \Gamma(1+\frac{1}{2}) \exp \left( \frac{\sqrt{2}\sigma}{4}\right) ^2
\end{align}
Reference~\cite{haenggi2009interference} gives the PDF of cumulative interference $I$ suffered by device at the origin:
\begin{align}
	f_{I}(x) = \frac{p\lambda_{m}}{4} (\frac{\pi}{x})^{\frac{3}{2}} \exp(-\frac{\pi^4 p^2\lambda_{m}^2}{16x}), x >= 0
\end{align}
We can get the PDF of cumulative interference $I$ in the presence of Rayleigh fading and log-normal shadowing, just by scaling $\lambda_{m}$ as $ \lambda_{m} \exp(\frac{\sigma^2}{8})$:
\begin{align}
	f_{I}(x) = \frac{    p\lambda_{m}   \exp(\frac{\sigma^2}{8})  }{4} (\frac{\pi}{x})^{\frac{3}{2}} \exp(    -\frac{    \pi^4    p^2\lambda_{m}^2     \exp^2(\frac{\sigma^2}{8})    }{    16x    }    ), 
\end{align} 
Hence,
\begin{align}
	\mathbb{E}\left[ I ^{-\frac{1}{2}}\right] &= \int_{0}^{+\infty} x^{-\frac{1}{2}} f_{I}(x) dx \nonumber\\
	&= \frac{4}{    \pi^{\frac{5}{2}}  p\lambda_{m} \exp(\frac{\sigma^2}{8}) }
\end{align}

\begin{align}
	\mathbb{E}\left[ \epsilon ^{\frac{1}{2}} \right] & = 4\Gamma(\frac{3}{2})\pi^{-\frac{5}{2}} p^{-1}\lambda_{m}^{-1}\nonumber\\
	&=2\pi^{-2} p^{-1}\lambda_{m}^{-1}
\end{align}
Therefore:
\begin{align}
	\mathcal{L}_{\Theta}\left( s \right) &= \exp(-\lambda_{b} \pi 2\pi^{-2}\lambda_{m}^{-1} \Gamma(\frac{1}{2}) s^{\frac{1}{2}}) \nonumber \\
	&= \exp(-2\lambda_{b} p^{-1}\lambda_{m}^{-1}\pi^{-\frac{1}{2}}  s^{\frac{1}{2}}) 
\end{align}
Its CDF and CCDF:
\begin{align}
	F_{\Theta} (x) &=1 -\erf \left( 
	\frac{\lambda_{b}p^{-1}\lambda_{m}^{-1}}{\sqrt{\pi x}}
	\right) \\
	F^c_{\Theta} (x) &=\erf(\frac{\lambda_{b} p^{-1}\lambda_{m}^{-1}}{\sqrt{\pi x}}) \nonumber\\
	&= \erf(\frac{1}{\frac{p\lambda_{m}}{\lambda_{b}}\sqrt{\pi x}})
\end{align}

The packet loss rate performance in the case of applying maximum ratio combining is illustrated in Fig.~\ref{fig:newpacketlossratemprmrc}. It is not surprisingly to observe that maximum ratio combing bring significant performance improve.
\begin{figure}
	\centering
	\includegraphics[width=1.0\columnwidth]{/Users/qsong/Documents/slotted_aloha_related_project/multiple_recepteur_capacity/new_packet_loss_rate_mpr_mrc}
	\caption{Network packet loss rate with respect to normalized load $p\lambda_{m}/\lambda_{b}$ (ANA=analytical, SIM=simulation)}
	\label{fig:newpacketlossratemprmrc}
\end{figure}

%Macro diversity gain against the nearest BS attach method $G_{\text{diversity}, n}$ is obtained by multiplying $\exp \left( \frac{\sqrt{2}\sigma}{\gamma}\right)^2$ to $(\ref{eq:macro-diversity-gain})$.
% depends on path-loss exponent $\gamma $, shadowing standard deviation $\sigma$ and packet loss rate constraint $P_f^{\text{max}}$, but has nothing to do with capture ratio.


%Old expression for P_f1
%\begin{align}
%\label{eq:bs_nst_att_analytical}
%P_{f1} &\approx 1 -\frac{1}{1 + \exp\left\lbrace \left( 1 +\frac{\pi \sigma_X^2}{8} \right)^{-\frac{1}{2}} \ln(B) \right\rbrace} \nonumber\\
%&=1-\frac{1}{1 + \left(B\frac{p\lambda_m}{\lambda_b} \right) ^{\left( 1 +\frac{\pi \sigma^2}{2\gamma^2} \right)^{-\frac{1}{2}}}}, \\
%L_{\text{max}} &= \frac{\left( \frac{P_{f1}^{\text{max}}}{1-P_{f1}^{\text{max}}}\right)^{\left( 1 +\frac{\pi \sigma^2}{2\gamma^2} \right)^{\frac{1}{2}}} }{B},
%\end{align}

%&= \int_{-\infty}^{+\infty} \frac{1}{ e^t+1}\frac{1}{\sqrt{2\pi} \frac{2}{\gamma}\sigma} \exp\left\lbrace -\frac{1}{2} \frac{(t-\ln(\frac{A}{\pi \lambda_{b}}))^2}{\left( \frac{2}{\gamma}\sigma\right)^2}\right\rbrace dt.

%\begin{align}
%\label{eq:mean_bx+1_step1}
%&P_{f}= \mathbb{E}_{r}\left[ 1-p_{s}\left(r\right) \right]  \nonumber\\
%&= 1-\int_{0}^{+\infty} \mathbb{E}_{G}\left[ \exp(-p\lambda_{m}AK r^2 e^{-\frac{2}{\gamma}G})\right]  2 \pi \lambda_b  r e^{-\lambda_b \pi r^2} dr \nonumber\\
%&= 1-\pi \lambda_b \mathbb{E}_{G}\left[ \int_{0}^{+\infty} \exp(-p\lambda_{m} A K r^2 e^{-\frac{2}{\gamma}G}-\lambda_b \pi r^2)\right] dr^2 \nonumber\\
%&= 1 -\mathbb{E}_{G}\left[ \frac{1}{\frac{AKp\lambda_{m}}{\pi \lambda_{b}}e^{-\frac{2}{\gamma}G}+1} \right] \nonumber\\
%&\mathbb{E}_{G}\left[ \frac{1}{\frac{p\lambda_{m}AK}{\pi \lambda_{b}}e^{-\frac{2}{\gamma}G}+1} \right]  = \int_{-\infty}^{+\infty} \frac{1}{ \frac{p\lambda_{m}AK}{\pi \lambda_{b}}e^t+1}\frac{\exp\left\lbrace -\frac{t^2}{2 \left( \frac{2}{\gamma}\sigma\right) ^2}\right\rbrace }{\sqrt{2\pi} \frac{2}{\gamma}\sigma} dt \nonumber\\
%&= \int_{-\infty}^{+\infty} \frac{\gamma}{ 2\sqrt{2\pi}\left( e^t+1\right)\sigma}e^{\left\lbrace -\frac{\gamma^2}{8 \sigma^2} (t-\ln\frac{AKp\lambda_{m}}{\pi \lambda_{b}})^2\right\rbrace} dt.
%\end{align}

%$\lim\limits_{\frac{p\lambda_{m}}{\lambda_{b} } +\infty} \frac{p\lambda_{m}}{\lambda_{b}}(1-P_{f2}) =\frac{1}{\frac{K}{\pi}\Gamma(1+\frac{2}{\gamma}) \theta_{T}^{\frac{2}{\gamma}}}$

%The normalized spatial throughput $S$ converges to a limit value $ \left[ K\Gamma(1+\frac{2}{\gamma})\theta_{T}^{\frac{2}{\gamma}} \right] ^{-1}\pi$, which depends on SINR threshold and path loss exponent.

%The analysis in Sec.~\ref{secsec:nst_bs_att} related to normalized spatial throughput still holds for this case.

%With $\ref{eq:bs_nst_att_max_load}$, we note that normalized spatial throughput $S$ converges to $(AK)^{-1}\pi$. It is worth indicating that normalized spatial throughput diverges to infinity when normalized load increases. However, this divergence is caused by the approximation in $(\ref{eq:bs_nst_att_analytical})$.

%Surely, the best BS attach method always outperforms than the nearest BS attach method. However, we still first consider the nearest BS attach method due to three reasons:\begin{inparaenum}[1)]
%\item nearest BS attach is a widely studied method in stochastic geometry field;
%\item nearest BS attach is the only choice when the downlink is not available (or seldom used for cost purpose) for some cheap systems;
%\end{inparaenum} 



%As discussed before, $(\ref{eq:bs_nst_att_analytical})$ is applicable for best BS attach method. To obtain a general result for both methods, we redefine $A$ is redefined as follows:
%\[ A=
%\begin{cases} 
%\Gamma(1+\frac{2}{\gamma}) \exp \left( \frac{\sqrt{2}\sigma}{\gamma}\right) ^2 \theta_{T}^{\frac{2}{\gamma}} , & \text{for Nearest BS attach }\\
%\Gamma(1+\frac{2}{\gamma})\theta_{T}^{\frac{2}{\gamma}} ,  & \parbox[t]{.4\columnwidth}{for Best BS attach}
%\end{cases}
%\]



%\begin{align}
%x & \\
%&\overset{\mathclap{\strut\text{a}}}= foo\\
%& = bar
%\end{align}