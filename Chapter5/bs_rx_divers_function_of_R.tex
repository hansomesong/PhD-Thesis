\section{Link Level Performance with outage probability constraint}
In Sec~\ref{sec:op_over_infinite_plane} and~\ref{sec:op_over_infinite_plane_with_retransmission}, we study the network-level performance metrics such as packet loss rate, throughput. These metrics are actually the averaged over the entire networks, which give a global view about the quality of service in LPWAN networks but still have its limitations. With a target network level packet loss rate such as $10\%$, it is very possible that devices in the neighborhood of BS never fail while devices at the border of BS coverage actually packet loss rate higher than $10\%$.

In this section, we propose a new performance indicator, link-level packet loss rate under the outage probability constraint. Concretely, the distance from the typical device to its best BS is known. The outage probability heres refers to the portion of devices whose link-level packet loss rate is superior to the target value. For example, a LPWAN should guarantee that at most $10\%$ of devices has packet loss rate higher than $1\%$.  We analyze both BS attach method and macro reception diversity with this proposed performance metric, and then evaluate the maximum normalized load, in one-shot random access case.

For a typical device at the origin, we assume that the best BS station with label $y_0$ is $r_{0}$ (in a transformed PPP when the shadowing is present) far away from the typical device. Recall that the distribution of $r_0$ is as follows:
\begin{align}
\label{eq:pdf_modified_r_0}
f(r) = 2 \pi \lambda_b e^{\frac{2\sigma^2}{\gamma^2}}  \exp( -\lambda_b  e^{\frac{2\sigma^2}{\gamma^2}} \pi r^2 ) r 
\end{align}







Equivalent Radius $R_{\text{eq}}$. We consider normalized distance by $R_{\text{eq}}$.
\begin{align}
	\pi R_{\text{eq}} ^ 2 = \frac{1}{\lambda_{b}} \\
	 R_{\text{eq}}  = \frac{1}{\sqrt{\pi \lambda_{b}}}
\end{align}
\subsection{Best BS Attach}
\begin{align}
	\label{eq:link-level-packet_loss_rate}
	P_{f, b}(r_{0}) = 1- \exp(-p \lambda_{m} \pi A \theta_{T}^{\frac{2}{\gamma}} e^{\frac{2\sigma^2}{\gamma^2}}  r_{0}^2 )
%	P_{f, b}(r_{0}^{'}) = 1- \exp(-A L \theta_{T}^{\frac{2}{\gamma}} e^{\frac{2\sigma^2}{\gamma^2}}  r_{0}^{'2} )
\end{align}
From $\eqref{eq:pdf_modified_r_0}$ and $\eqref{eq:link-level-packet_loss_rate}$, the probability density function of packet loss rate $P_{f, b}(r_{0})$ is obtained as follows:
\begin{align}
	f_{P_{f,b}}(p_{f,b}) =\frac{1}{ A \theta^{\frac{2}{\gamma}} L } (1-p_f)^{ 1 / \left( A \theta^{\frac{2}{\gamma}} L\right) - 1} 
\end{align}
The cumulative distribution function of $P_{f,b}$:
\begin{align}
F_{P_{f,b}}(p_{f,b}) =1 - (1-p_f)^{ 1 / \left( A \theta^{\frac{2}{\gamma}} L\right) } 
\end{align}
According to the outage probability definition, $P_{\text{outage}, b}$ is as follows:
\begin{align}
	P_{\text{outage}, b} = (1-p_f)^{ 1 / \left( A \theta^{\frac{2}{\gamma}} L \right) } 
\end{align}

\subsection{Macro Reception Diversity}
Recall that in case of macro reception diversity, 
\begin{align}
\label{eq:definition_pf1} 
P_{f, m}(r_{0}) &= \mathbb{P}\left\lbrace  \theta_{y_b} < \theta_{T}, \theta_{y_b} < \theta_{T} \text{where} \right\rbrace \nonumber\\
&= ( 1-p_s( r_{0} ) ) \mathbb{E} \left[  \prod_{r_i \in \Phi_{b}} (1-p_{s}(r_i)) \right], \text{with } r_i \in \left[ r_{0}, +\infty\right],
\end{align} 

where $r_i$ refers to the distance between the device and BS with label $i$. Note that $r_i$ should be not less than distance to the best BS denoted by $r_{0}$, whose probability density function is given in $\eqref{eq:pdf_nearest_distance}$. Expression $\eqref{eq:definition_pf1}$ is actually the Probability Generating FunctionaL (PGFL) of Point point process $\Phi_{b}$ for Base Station, which states for some function $f(x)$ that $\mathbb{E}\left[ \prod_{x \in \Phi}f(x) \right] = \exp(-\lambda(\int_{\mathbb{R}^2}(1-f(x))dx))$. Thus, term in $\mathbb{E} \left[  \prod_{r_i \in \Phi_{b}} (1-p_{s}(r_i)) \right]$ in $\eqref{eq:definition_pf1}$ can be expressed as follows:
\begin{align}
\mathbb{E} \left[  \prod_{r_i \in \Phi_{b}} (1-p_{s}(r_i)) \right] &= \exp\left\lbrace -2\pi \lambda_{b} \int_{r_{0}}^{+\infty} p_{s}(r)rdr \right\rbrace \nonumber\\
&= \exp\left\lbrace  \int_{r_{0}}^{+\infty}  \exp(-p \lambda_{m} \pi A \theta_{T}^{\frac{2}{\gamma}} e^{\frac{2\sigma^2}{\gamma^2}}  r^2) 2 \pi \lambda_b e^{\frac{2\sigma^2}{\gamma^2}}  rdr  \right\rbrace \nonumber \\ 
&= \exp\left\lbrace -\pi \lambda_{b} e^{\frac{2\sigma^2}{\gamma^2}}  \int_{r_{0}^{2}}^{+\infty}  \exp(-p \lambda_{m} \pi A \theta_{T}^{\frac{2}{\gamma}} e^{\frac{2\sigma^2}{\gamma^2}}  r^2 ) dr^2  \right\rbrace \nonumber\\ 
&= \exp\left\lbrace -\frac{1}{ A \theta_{T}^{\frac{2}{\gamma}}  L }  \exp(-p \lambda_{m} \pi A \theta_{T}^{\frac{2}{\gamma}} e^{  \frac{2\sigma^2}{\gamma^2}  } r_{0}^2)  \right\rbrace
\end{align} 

Thus,
\begin{align}
P_{f}(r_{0}) = ( 1- \exp(-p \lambda_{m} \pi A \theta_{T}^{\frac{2}{\gamma}} e^{\frac{2\sigma^2}{\gamma^2}}  r_{0}^2 ) )\exp\left\lbrace \!\!\!-\frac{1}{ A \theta_{T}^{\frac{2}{\gamma}}  L }  \!\exp(-p \lambda_{m} \pi A \theta_{T}^{\frac{2}{\gamma}} e^{  \frac{2\sigma^2}{\gamma^2}  } r_{0}^2)  \!\!\!\right\rbrace
\end{align}


Outage probability:
\begin{align}
	P_{\text{outage}} &= \int_{0}^{+\infty} \mathds{1}_{\left[ P_{f, m} > P_{f}^{\text{max}}\right] }f(r) dr \nonumber \\
	&= \int_{R}^{+\infty} f(r) dr \nonumber \\
	&= 1 - \int_{ 0 }^{ R } f(r) dr \nonumber \\
	&=  \exp( -\lambda_b  e^{\frac{2\sigma^2}{\gamma^2}} \pi R^2 ),
\end{align}
where $R$ is the minimum distance between typical device and BS $y_0$ satisfying condition $P_{f, m} > P_{f}^{\text{max}}$.
\qsong{To be further simplified...}


\subsection{Impact of background noise to minimum transmit power}
We assume that a network set a target packet loss rate $P_f$, for example $10\%$. The objective is that at most $P_{outage}$ percent of devices whose packet loss rate is superior to $P_f$. Thus, $P_{outage}$ is referred as to the outage probability.
If we set the normalized background noise $N$ be $1$, we want to the study the relationship between the device minimum transmit power $p_t$ and the outage probability $P_{outage}$.

Now let $p_t$ the device transmit power. Still $p_r$ is the received power at BS side.  The SNR of a packet transmission $\theta$ is:
\begin{align}
	\theta &= \frac{P_{r}}{N} \\
	&= \frac{p_t H r^{-\gamma}}{N}
\end{align}
The transmission success probability $P_{s}$:
\begin{align}
	P_{s} = \exp( -\frac{ \theta_{T} r^{\gamma}}{p_{t}}).
\end{align}
Actually, $P_{s}$ is a function of RV $r$. To calculate the outage probability, we need to know the cumulative distribution function (CDF) $F_{P_{s}}$ of RV $P_{s}$. 
%\begin{align}
%	f_{P_{s}}\left( x \right) = 2\pi\lambda_{b} \left[ \frac{1}{A} \log(\frac{1}{x})\right] ^{\frac{1}{\gamma}}\exp\left\lbrace  -\pi\lambda_{b} \left[  \frac{1}{A} \log(\frac{1}{x})\right]  ^{\frac{2}{\gamma}} \right\rbrace \frac{1}{\gamma} \left[ \frac{1}{A} \log(\frac{1}{x})\right] ^{\frac{1}{\gamma} - 1 } \frac{1}{Ax}
%\end{align}

With packet loss rate target $P_f$, the outage probability $P_{\text{outage}}$
\begin{align}
	\label{eq:p_outage_vs_min_tx_power}
	P_{\text{outage}} &= F_{P_{s}}\left( 1- P_{f} \right)  \nonumber\\
	&= \mathbb{P} \left\lbrace  \theta \leq  1- P_{f}  \right\rbrace  \nonumber\\
	&=  1  -  \mathbb{P} \left\lbrace  r \leq  \left[ \frac{p_t}{\theta_{T}} \log(\frac{1}{ 1- P_{f} })\right] ^{\frac{1}{\gamma}} \right\rbrace \nonumber\\
	&= 1 -F_r \left( \left[ \frac{p_t}{\theta_{T}} \log(\frac{1}{ 1- P_{f} })\right] ^{\frac{1}{\gamma}} \right) \nonumber\\
	&= \exp\left\lbrace -\pi \lambda_{b} \exp\left( \frac{2\sigma^2}{\gamma^2}\right) \left[  \frac{p_t}{\theta_{T}} \log(\frac{1}{ 1- P_{f} })\right]  ^{\frac{2}{\gamma}} \right\rbrace 
\end{align}
Hence, by inversing $\eqref{eq:p_outage_vs_min_tx_power}$, the minimum transmit power $p_t$ is:
\begin{align}
	\label{eq:min_tx_power_vs_p_outage_plr}
	p_t = \left[ \frac{1}{\pi \lambda_{b}} \ln\left( 1/P_{\text{outage}} \right) \right] ^{\frac{\gamma}{2}} \frac{\theta_{T}} {\ln(1/(1-P_{f}))}
\end{align}


\subsubsection{Macro reception diversity}
Actually, the packet loss rate is a function of a realization of BS location.
\begin{align}
	\mathbb{E} \left[  \prod_{r_i \in \Phi_{b}} (1-p_{s}(r_i)) \right] &= \exp\left\lbrace -2\pi \lambda_{b} \int_{r_{0}}^{+\infty} p_{s}(r)rdr \right\rbrace \nonumber\\
	&= \exp\left\lbrace  \int_{r_{0}}^{+\infty}  \exp( -\frac{N \theta_{T} r^{\gamma}}{P_{r}}) 2 \pi \lambda_b e^{\frac{2\sigma^2}{\gamma^2}}  rdr  \right\rbrace \nonumber \\ 
	&= \exp\left\lbrace -\pi \lambda_{b} e^{\frac{2\sigma^2}{\gamma^2}}  \int_{r_{0}^{2}}^{+\infty}  \exp( -\frac{N \theta_{T} r^{\gamma}}{P_{r}}) dr^2  \right\rbrace 
\end{align} 
Only when $\gamma = 4$, the closed-form expression exists:
\begin{align}
	\mathbb{E} \left[  \prod_{r_i \in \Phi_{b}} (1-p_{s}(r_i)) \right] &= \exp\left\lbrace -\pi \lambda_{b} e^{\frac{2\sigma^2}{\gamma^2}}  \frac{1}{\sqrt{A}} \erfc\left( \sqrt{A} r_0^2 \right)   \right\rbrace 
\end{align} 

Thus,
\begin{align}
	P_{f}(r_{0}) = ( 1- \exp( -\frac{N \theta_{T} r^{\gamma}}{P_{r}}) )
	\exp\left\lbrace -\pi \lambda_{b} e^{\frac{2\sigma^2}{\gamma^2}}  \frac{1}{\sqrt{A}} \erfc\left( \sqrt{A} r_0^2 \right)   \right\rbrace 
\end{align}


Outage probability:
\begin{align}
	P_{\text{outage}} &= \int_{0}^{+\infty} \mathds{1}_{\left[ P_{f, m} > P_{f}^{\text{max}}\right] }f(r) dr \nonumber \\
	&= \int_{R}^{+\infty} f(r) dr \nonumber \\
	&= 1 - \int_{ 0 }^{ R } f(r) dr \nonumber \\
	&=  \exp( -\lambda_b  e^{\frac{2\sigma^2}{\gamma^2}} \pi R^2 ),
\end{align}
where $R$ is the minimum distance between typical device and BS $y_0$ satisfying condition $P_{f, m} > P_{f}^{\text{max}}$.
\qsong{To be further simplified...}