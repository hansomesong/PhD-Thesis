\section{Conclusion}
\label{sec:conclusion}
%The contributions and organization of this letter are as follows: Sec.~\ref{sec:system_model} introduces the system model commonly used in the literature, which takes into account Rayleigh fading, shadowing and capture effect in wireless channel. Sec.~\ref{sec:trans_succ_p_pair} synthesizes some stochastic geometry research results useful for the subsequent analysis, and proposes an upper bound for maximum interference analysis. Sec.~\ref{sec:op_over_infinite_plane} presents our analysis for pure (i.e. non-slotted) and slotted ALOHA. It gives closed-form expressions, which are simple but accurate enough to quantify the macro reception diversity gain for a given packet loss rate objective. Sec.~\ref{sec:simulation} presents simulation results and analysis. Sec.~\ref{sec:conclusion} concludes this letter.
In this letter, we derive closed-form expressions for the network packet loss rate and macro diversity gain in pure and slotted ALOHA wireless networks taking into account Rayleigh fading, shadowing and capture effect. We find that systems with macro diversity benefit from shadowing effect and provide an important gain in all cases. 
%In addition, we observe that macro diversity gain depends on path-loss exponent, shadowing standard deviation and packet loss rate constraint, and does not depend on capture ratio and ALOHA types.