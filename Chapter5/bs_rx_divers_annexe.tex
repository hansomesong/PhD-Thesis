%\section*{Appendices}
\addcontentsline{toc}{section}{Appendices}
\renewcommand{\thesubsection}{\Alph{subsection}}
\label{sec:annexe}
Recall that $r_{x_i}$ is the distance between a BS at the origin and one given device $x_i$, $\theta$ is the SINR threshold for capture effect. $H_{x_i}$ and $\exp(G_{x_i})$ are respectively random variable accounting for Rayleigh fading and lognormal shadowing.
\begin{align}
p_{s} \left( r_{x_i} \right)
%&= \mathbb{P}\left\lbrace \frac{P_r}{I^{'}} \geq \theta \right\rbrace \nonumber\\
& =\mathbb{P}\left\lbrace \frac{H_{x_i} \exp(G_{x_i}) r_{x_i}^{-\gamma}}{\sum_{x_j \in \Phi_{m}} H_{, x_j} \exp(G_{x_j}) r_{x_j}^{-\gamma}}  \geq \theta \right\rbrace \nonumber\\
%& =\mathbb{P} \left\lbrace \frac{H_{x_i}  r_{x_i}^{-\gamma}}{\sum_{x_j \in \Phi_{m}} H_{x_j} \exp(G_{x_j}-G_{x_i}) r_{x_j}^{-\gamma}} \geq \theta \right\rbrace \nonumber
\end{align}
Let $I=\sum_{x_j \in \Phi_{m}} H_{x_j} \exp(G_{x_j}) r_{x_j}^{-\gamma}$, which is the cumulative interference suffered for device $x_i$. Thus, we have:
%Let $Z_{3}= \exp(G_{x_j}-G_{x_i})$, $H = Z_1 Z_3$. $I$ be 
\begin{align}
\label{eq:def_ps}
p_{s}\left( r \right)  &= Pr \left\lbrace H_{x_i} \geq I \theta\exp(-G_{x_i})r ^{\gamma}  \right\rbrace  \nonumber\\
&=\mathbb{E}_{G_{x_i}}\left\lbrace  \mathbb{E}_{I} \left[ \exp(-\theta \exp(-g_{x_i}) r^{\gamma}  I ) \vert G_{x_i} = g_{x_i}\right]\right\rbrace   \nonumber\\
&= \mathbb{E}_{G_{x_i}} \left[ \mathcal{L}_{I}\left\lbrace \theta \exp(-G_{x_i}) r^{\gamma}\right\rbrace \right] 
\end{align}
It is observed that the $p_s\left( r \right)$ is actually the expectation (w.r.t $G_{x_i}$) of a conditional Laplace Transform of cumulative interference $I$ at point $\theta \exp(-G_{x_i})  r_{x_i}^{\gamma}$.
\begin{align}
\label{eq:interferece-laplace-transform}
\mathcal{L}_{I}\left( s \right)  &= \mathbb{E}\left[ \exp(-sI)\right] \nonumber\\
&= \mathbb{E}\left[ \exp(-s\sum_{x_j \in \Phi_m} H_{x_j} \exp(G_{x_j})  r_{x_j}^{-\gamma})\right] \nonumber \\
&= \mathbb{E}_{\Phi_m} \left\lbrace \prod_{x_j\in \Phi_m} \mathbb{E}\left[\exp(-sH_{x_j} \exp(G_{x_j})  r_{x_j}^{-\gamma})\right] \right\rbrace 
\end{align}

Formula (\ref{eq:interferece-laplace-transform}) is actually the Probability Generating Functional (PGFL) of a Poisson point process.
Applying Campell's theorem, 
%\qsong{Here I give the wikipedia link for this theorem, in the final version, I will give a reference of one book: https://en.wikipedia.org/wiki/Campbell's_theorem_(probability)}
\begin{align}
\label{eq:laplace_trans_I}
\mathcal{L}_{I}\left( s \right) &=\exp\left\lbrace -\lambda_m \pi \mathbb{E}\left[ H^{\frac{2}{\gamma}}\right]\mathbb{E}\left[ \exp(\frac{2}{\gamma}G_{x_j})\right] \Gamma(1-\frac{2}{\gamma})s^{\frac{2}{\gamma}}  \right\rbrace 
\end{align}
This result has been derived in~\cite[eq.3.20]{haenggi2009interference}.

\begin{align}
\label{eq:lognormal_moment}
\mathbb{E}\left[ \exp(\frac{2}{\gamma}G_{x_j})\right] &= \exp \left\lbrace \left( \frac{\sqrt{2}\beta\sigma}{\gamma}\right) ^2\right\rbrace 
\end{align}
\begin{align}
\label{eq:expo_moment}
\mathbb{E}\left[ H^{\frac{2}{\gamma}} \right] &= \Gamma(1+\frac{2}{\gamma}) 
\end{align}

Substituting $(\ref{eq:expo_moment})(\ref{eq:lognormal_moment})$ into $(\ref{eq:laplace_trans_I})$, we have:
\begin{align}
\mathcal{L}_{I}\left( s \right) &= \exp\left\lbrace -p\lambda_m \pi \Gamma(1+\frac{2}{\gamma}) \Gamma(1-\frac{2}{\gamma}) \exp \left( \frac{\sqrt{2}\beta\sigma}{\gamma}\right) ^2 s^{\frac{2}{\gamma}}  \right\rbrace
\end{align}
Let $s=\theta \exp(-G) r^{\gamma}$, thus
\begin{align}
	\mathcal{L}_{I}\left\lbrace \theta \exp(-G) r^{\gamma}\right\rbrace = \exp\left\lbrace -A r^2\exp(-\frac{2}{\gamma}G)\right\rbrace ,
\end{align}
where $A=p\lambda_m\pi \Gamma(1+\frac{2}{\gamma}) \Gamma(1-\frac{2}{\gamma}) \exp \left( \frac{\sqrt{2}\beta\sigma}{\gamma}\right) ^2 \theta^{\frac{2}{\gamma}}$.


We continue to process formula $(\ref{eq:def_ps})$. It remains to calculate the expectation with respect to $G$:
\begin{align}
	\label{eq:def_ps_2}
	p_{s}(r) &= \mathbb{E}_{G}\left[ \exp(-A r^2 \exp(-\frac{2}{\gamma}G)) \right] \nonumber\\
	&=\int_{-\infty}^{+\infty} \exp(-A r^2 e^{-\frac{2}{\gamma}x})\frac{1}{\sqrt{2\pi}\sigma} \exp(-\frac{x^2}{2\sigma^2}) dx \nonumber\\
	&=\int_{-\infty}^{+\infty} \exp(-A r^2 e^{x})\frac{1}{\sqrt{2\pi}\frac{2}{\gamma}\sigma} \exp(-\frac{x^2}{2(\frac{2}{\gamma})^2\sigma^2}) dx 
\end{align}
Formula $(\ref{eq:def_ps_2})$ is actually the value of Laplace Transform for one log-normal random variable $X \sim LN(0, \frac{2}{\gamma}\sigma)$ at point $Ar^2$.
\begin{align}
	p_{s}(r) &= \mathbb{E}\left[ \exp(-Ar^2X)\right] \nonumber\\
	&= \mathcal{L}_{X} \left[ Ar^2\right]  
\end{align}
Now we will not consider to get the closed-form expression for $p_{s}(r)$, because in the following the form will simplify the analysis. 
%现在我们先不考虑:获取P_s的解析解,因为在下一步计算会更方便
%A closed form expression of the Laplace transform of the lognormal distribution does not
%exist. According to reference~\cite{asmussen2016laplace}, the Laplace transform of a log-normal random variable can be accurately approximated as follows:
%\begin{align}
%\label{eq:laplace-transform-lognormal-form-1}
%\mathcal{L} \left\lbrace X \right\rbrace \left( s \right)
%&= \frac{\exp(-\frac{W(s \sigma_{X}^2 e^{\mu_{X}} )^2 + 2W(s \sigma_{X}^2 e^{\mu_{X}})}{2\sigma_{X}^2})}{\sqrt{1 + W(s \sigma_{X}^2 e^{\mu_{X}})}},
%\end{align}
%where $W\left( \cdot \right)$ is the Lambert W function~\cite{corless1996lambertw}, which is defined as the solution in principal branch of the
%equation $W\left(x\right) e^{W \left( x\right) }= x$.

We want to know the outage probability $P_{f1}$ over an infinite plane. Our strategy is to start with outage probability $P_{f1}(R)$ over a finite circle region with radius $R$, then extend to an infinite area. We still assume one device is located at the origin of the circle region with radius $R$. The spatial density of base station is $\pi R^2 \lambda_b$. The number of base station $N$ is thus:
\begin{align}
	\mathbb{P}\left[ N=n\right] = e^{-\pi R^2 \lambda_b}\frac{\left( \pi R^2 \lambda_b\right) ^n}{n!} 
\end{align}
Since the base station is uniformly distributed in the circle region, the probability density function of distance between a given base station and the device $r$ at the origin is as follows:
\begin{align}
	f\left(r\right) = \frac{2r}{R^2}, r \in \left[ 0, R\right]  
\end{align} 

Thus, the outage probability $P_f\left( R\right) $ conditioned on base station $N$ is as follows:
\begin{align}
	P_{f1}\left( R \vert N= n\right ) &=  \prod_{i=1}^{n}\int_{0}^{R}(1-p_{s}(r_i))f(r_i) dr_i \nonumber\\
	&= \left[ 1-\int_{0}^{R} p_s\left(r\right)f\left( r\right)dr \right]^n 
\end{align} 

Next step, we decondition for  $P_{f1}\left( R \vert N= n\right )$, we have the expression $P_{f1}\left( R \right )$, which is actually the probability generating function (PGF) of Poisson distribution with intensity $\pi R^2 \lambda_b$:
\begin{align}
		P_{f1}\left( R\right ) &= \mathbb{E}\left[ P_f\left( R \vert N= n\right )  \right] \nonumber\\
		&= \exp(-\pi R^2 \lambda_b \int_{0}^{R} p_{s}\left( r \right)f\left( r\right)dr  ) \\
		&= \exp(-\pi R^2 \lambda_b \int_{0}^{R} \mathbb{E}\left[ e^{-Ar^2X} \right]  \frac{2r}{R^2}dr) \\ 
		&= \exp(-\pi R^2 \lambda_b \mathbb{E}\left[\int_{0}^{R} e^{-Ar^2X} \frac{2r}{R^2}dr \right] ) \\ 
		&= \exp(-\pi R^2 \lambda_b \mathbb{E}\left[\int_{0}^{R} e^{-Ar^2X} \frac{2r}{R^2}dr \right] ) \\ 
		&= \exp(-\pi\lambda_b \mathbb{E}\left[\int_{0}^{R^2} e^{-ArX} dr \right] ) \\
		&= \exp( \pi\lambda_b \mathbb{E}\left[ \frac{e^{-ArX} }{AX} \vert_{0}^{R^2}\right] ) \\
		&= \exp( \pi\lambda_b \mathbb{E}\left[ \frac{e^{-AR^2X} }{AX} - \frac{1}{AX}\right] )
\end{align}
By extending $R$ to $+\infty$, we get the outage probability over infinite plane:
\begin{align}
	\label{eq:analytical_result_approach_1}
	P_f &=  \exp( \pi\lambda_b A \mathbb{E}\left[ {X^{-1}}\right] ) \\
	&= \exp(-\frac{1}{\frac{p\lambda_m}{\lambda_b}\Gamma(1+\frac{2}{\gamma}) \Gamma(1-\frac{2}{\gamma})\theta^{\frac{2}{\gamma}}} )
\end{align}
We surprisingly observe that with BS reception diversity, the limit case, the outage probability has the same formula with the case where just Rayleigh fading is considered. But in practical system, we still observe that shadowing has improved the outage probability, since the wireless network cannot be infinite and the population is always limited. 


As a comparison reference, now we consider probability $P_{f2}$ over an infinite plane using traditional method: the device attach the nearest BS.
The distance to the nearest base station, we denote this distance as $r$. Its PDF is:
\begin{align}
f\left( r\right)  = 2 \pi \lambda_b  r \exp(-\lambda_b \pi r^2), r \in \left[ 0, +\infty\right] 
\end{align}
\begin{align}
P_{f2}\left( R \right) &= \mathbb{E}\left[ 1-p_{s}\left(r\right) \right]  \nonumber\\
&= 1-\int_{0}^{R}   p_{s}\left(r\right)  f\left( r\right) dr \nonumber\\
&= 1-\int_{0}^{R} \mathbb{E}\left[ \exp(-A r^2 X)\right]  2 \pi \lambda_b  r \exp(-\lambda_b \pi r^2) dr \nonumber\\
&= 1-  \pi \lambda_b \mathbb{E}\left[ \int_{0}^{R} \exp\left\lbrace -\left( AX+\lambda_b \pi\right)  r^2 \right\rbrace    dr^2 \right] \nonumber\\
&= 1 + \frac{\pi \lambda_b \exp\left\lbrace -\left( AX+\lambda_b \pi\right)  r^2 \right\rbrace }{AX+\lambda_b \pi}\vert_{0}^{R} 
\end{align}
Similarly, by extending $R$ to $+\infty$, we have $P_{f2}$: 
\begin{align}
P_{f2} &= 1 - \mathbb{E}\left[ \frac{1}{\frac{AX}{\pi\lambda_b}+1} \right] \nonumber
\end{align}
For the simplicity  of writing in the following, let $B= A \left( \pi \lambda \right) ^{-1}$. And we will focus on the term $\mathbb{E}\left[ \frac{1}{BX+1} \right]$. Thus, 
\begin{align}
\label{eq:mean_bx+1_step1}
&\mathbb{E}\left[ \frac{1}{BX+1} \right] =  \int_{0}^{+\infty} \frac{1}{\left( Bx+1\right) \sqrt{2\pi}\sigma_X x} \exp(-\frac{(\ln(x))^2}{2\sigma_X^2})dx\nonumber\\
&= \int_{-\infty}^{+\infty} \frac{1}{ Be^t+1} \cdot \frac{1}{\sqrt{2\pi} \sigma_X} \exp\left\lbrace -\frac{t^2}{2 \sigma_X^2}\right\rbrace dt \nonumber\\
&= \int_{-\infty}^{+\infty} \frac{1}{1+e^{-(t-\ln(B))}} \cdot \frac{1}{\sqrt{2\pi} \sigma_X} \exp\left\lbrace -\frac{t^2}{2 \sigma_X^2}\right\rbrace dt \nonumber\\
&= \int_{-\infty}^{+\infty} \frac{1}{1+\exp{-\left( \frac{t-\frac{\ln(B)}{\sigma_X}}{\frac{1}{\sigma_X}}\right) }} \cdot \frac{1}{\sqrt{2\pi}} \exp\left\lbrace -\frac{t^2}{2}\right\rbrace dt \nonumber\\
& = \int_{-\infty}^{+\infty} \frac{\phi\left( t \right) }{1+\exp{-\left( \frac{t-\frac{\ln(B)}{\sigma_X}}{\frac{1}{\sigma_X}}\right) }} dt,
\end{align}
where $\phi\left( t \right)$ is the PDF of standard normal distribution. According to~\cite{crooks2009logistic}, the logistic function can be approximately by an error function:
\begin{align}
\label{eq:logistical_erf_fun_appro}
\frac{1}{1 + \exp(-\frac{x}{\alpha})} &\approx \frac{1}{2} + \frac{1}{2}\erf(\frac{\sqrt{\pi}}{4\alpha}x),
\end{align}
where $\alpha$ is the parameter of logistic function. Applying $(\ref{eq:logistical_erf_fun_appro})$ for the denominator of integrand in $(\ref{eq:mean_bx+1_step1})$ , we have:
\begin{align}
\frac{1}{1+\exp{-\left( \frac{t-\frac{\ln(B)}{\sigma_X}}{\frac{1}{\sigma_X}}\right) }} &\approx \frac{1}{2} + \frac{1}{2}\erf(\frac{\sqrt{\pi}\sigma}{4}\left( t-\frac{\ln(B)}{\sigma_X}\right) ) \nonumber\\
&= \Phi\left( \sqrt{\frac{\pi}{8}}\sigma\left( t-\frac{\ln(B)}{\sigma_X}\right) \right) 
\end{align}
Therefore:
\begin{align}
&\mathbb{E}\left[ \frac{1}{BX+1} \right] = \int_{-\infty}^{+\infty} \Phi\left( \sqrt{\frac{\pi}{8}}\sigma_X\left( t-\frac{\ln(B)}{\sigma_X}\right) \right) \phi\left( t \right) dt \nonumber\\ 
&= Pr \left\lbrace X_1 \leq \sqrt{\frac{\pi}{8}}\sigma_X\left( X_2-\frac{\ln(B)}{\sigma}\right) \right\rbrace \nonumber\\
&= Pr \left\lbrace X_1 - \sqrt{\frac{\pi}{8}}\sigma_X X_2 \leq -\frac{\ln(B)}{2} \sqrt{\frac{\pi}{2}} \right\rbrace 
\end{align}
where $X_1, X_2$ are independent standard normal random variable. Obviously, $X_1 - \sqrt{\frac{\pi}{8}}\sigma_X X_2 \sim \mathcal{N}\left( 0,  1+ \frac{\pi}{8}\sigma_X^2\right) $. Hence,
\begin{align}
\label{eq:mean_bx+1_step2}
\mathbb{E}\left[ \frac{1}{BX+1} \right] &= Pr \left\lbrace \frac{X_1 - \sqrt{\frac{\pi}{8}}\sigma_X X_2}{\sqrt{1+ \frac{\pi}{8}\sigma_X^2}} \leq -\frac{\ln(B)}{\sqrt{\frac{8}{\pi}+ \sigma_X^2}}\right\rbrace  \nonumber\\
&= \Phi\left( -\left( \frac{8}{\pi}+ \sigma_X^2\right) ^{-\frac{1}{2}}\ln(B) \right) \nonumber\\
&= \frac{1}{2} + \frac{1}{2} \erf \left\lbrace -\left( \frac{16}{\pi}+ 2\sigma_X^2\right) ^{-\frac{1}{2}}\ln(B)\right\rbrace 
\end{align}
Still, applying $(\ref{eq:logistical_erf_fun_appro})$ for $(\ref{eq:mean_bx+1_step2})$, we have:
% 实际上此图可以删除
%\begin{figure}
%	\centering
%	\includegraphics[width=0.7\linewidth]{Figures/erf_approx}
%	\caption{The approximation of the logistic function by error function. The absolute difference between the two functions is always less than 0.02.}
%	\label{fig:erf_approx}
%\end{figure}
\begin{align}
\label{eq:analytical_result_approach_2}
P_{f2} &= 1 - \frac{1}{1 + \exp\left\lbrace \left( 1 +\frac{\pi \sigma_X^2}{8} \right)^{-\frac{1}{2}} \ln(B) \right\rbrace} \nonumber\\
&= 1-\frac{1}{1 + B^{\left( 1 +\frac{\pi \sigma_X^2}{8} \right)^{-\frac{1}{2}}}},
\end{align}
where $B= \frac{p\lambda_m}{\lambda_b}\Gamma(1+\frac{2}{\gamma}) \Gamma(1-\frac{2}{\gamma}) \exp \left( \frac{\sqrt{2}\beta\sigma}{\gamma}\right) ^2 \theta^{\frac{2}{\gamma}}$.