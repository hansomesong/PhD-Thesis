\section{Results and Analysis}
\label{sec:simulation}
\begin{figure*}
	%	\centering
	\subfigure[Slotted ALOHA]
	{  \label{fig:subfig:a} %% label for second subfigure
		\includegraphics[width=0.36\linewidth]{/Users/qsong/Documents/slotted_aloha_related_project/multiple_recepteur_capacity/new_packet_loss_rate_mpr.eps}}
	\hspace{-2.3em}
	\subfigure
	[Pure ALOHA, average interference]
	{
		\label{fig:subfig:b} %% label for first subfigure
		\includegraphics[width=0.36\linewidth]{/Users/qsong/Documents/slotted_aloha_related_project/multiple_recepteur_capacity/pure_slot_packet_loss_rate_mpr.eps}}
	%		\hspace{1in}
	\hspace{-2.1em}
	\subfigure
	[Pure ALOHA, maximal interference]
	{
		\label{fig:subfig:c} %% label for second subfigure
		\includegraphics[width=0.35\linewidth]{/Users/qsong/Documents/slotted_aloha_related_project/multiple_recepteur_capacity/analytical_model_validation/validation_plr}
	}
	\caption{Network packet loss rate with respect to normalized load $p\lambda_{m}/\lambda_{b}$ (ANA=analytical, SIM=simulation)}
	\label{fig:subfig} %% label for entire figure
	\vspace{-2em}
\end{figure*}
We develop a simulator in which devices and BS are deployed according to PPP. The interference on each BS is computed without any independence assumption. The basic simulation settings are: $\gamma=4$, $\sigma = 0$ or $8$ dB, $\theta_{T} =3$ dB, $p=0.008$, $P_{f}^{\text{max}}=10\%$. The device density $\lambda_m$ varies from $0.2$ to $2$, and BS density $\lambda_{b}$ is fixed as $0.08$. To calculate the $95\%$ confidence interval, each simulation is repeated $50$ times. 

Fig.~\ref{fig:subfig} shows the packet loss rate from both simulation and analytical results. For the nearest and best BS attach cases, the simulation and analytical results fit well:\begin{inparaenum}[i)]
	\item this confirms the correctness of $(\ref{eq:bs_nst_att_analytical})$ (which is an approximation) and $(\ref{eq:bs_best_att_analytical})$;
	\item the upper bound proposed in Sec.~\ref{subsec: pure-aloha} for maximum interference is a simple but accurate approximation (see Fig.~\ref{fig:subfig:c}).
\end{inparaenum} 
As proved by $(\ref{eq:bs_best_att_analytical})$, the nearest BS attach with $\sigma=0$ dB is identical to the best BS attach case. Our simulation confirms this, but we just plot the latter in Fig.~\ref{fig:subfig} for the sake of clarity. 


For macro diversity case, the packet loss rate in slotted ALOHA given by the analysis is lower than the rate obtained by simulations (see Fig.~\ref{fig:subfig:a}). The bias is larger without shadowing. The is due to the fact that the interference independence assumption does not always hold, especially at low load regime and without shadowing. When the load increases, diversity of interferes increases, and the deviation between simulation and analysis is reduced. In the presence of shadowing, the randomness of interferences increases and the same trend is observed. Thus, shadowing effect can be leveraged to reduce packet loss rate if macro reception diversity is applied. With pure ALOHA and average interference (see Fig.~\ref{fig:subfig:b}), the lack of synchronization also increases the randomness of interferences, and the deviation is smaller than that in slotted ALOHA. With pure ALOHA and maximum interference (see Fig.~\ref{fig:subfig:c}), the interference independence assumption and upper bound approximation jointly reduce the deviation.

Actually, in realistic deployed M2M-dedicated networks, especially in urban areas, the shadowing effect is present. In addition, LPWAN apply macro diversity and pure ALOHA~\cite{ietf-lpwan-overview-03}. The analytical model is relatively accurate around packet loss rate of $10\%$, which makes $(\ref{eq:bs_rx_divers_analytical})$ suitable to analyze this kind of networks.

From $(\ref{eq:bs_nst_att_analytical})$ and $(\ref{eq:bs_rx_divers_analytical})$, the maximum normalized loads in different scenario are obtained and shown in Tab.~\ref{my-label}. We observe that macro reception diversity supports about $4.0$ times devices as much as that served by best BS attach method in case of $8$ dB shadowing. 
In addition, the maximum load with macro diversity for pure ALOHA in the worst case is $0.1$, which is about $2$ times the maximum load ($0.05$) for slotted ALOHA in best case. This means that for very cheap systems in which devices do not manage any downlink channel, the lack of synchronization is more than compensated by the macro diversity gain.
\begin{table}[]
	\centering
	\caption{Maximum normalized load under packet loss rate $10\%$ and corresponding diversity gain with $8$ dB shadowing}
	\label{my-label}
	\resizebox{\columnwidth}{!}{%
		\begin{tabular}{@{}lccc@{}}
			\toprule
			& \multicolumn{1}{l}{Slotted ALOHA} & \multicolumn{1}{l}{\shortstack{Pure ALOHA\\Avg. Interference}} & \multicolumn{1}{l}{\shortstack{Pure ALOHA\\Max. Interference}} \\ \midrule
			Best BS Attach                                      & $\textbf{0.05}$                            & $0.038$                                                        & $0.025$                                                        \\ \midrule
			Macro Diversity                                     & $0.2$                             & $0.152$                                                        & $\textbf{0.1}$                                                          \\ \midrule
			Macro Diversity Gain & \multicolumn{3}{c}{$\times 4.0$}                                                                                                                                    \\ \bottomrule
		\end{tabular}%
	}
\end{table}

%\begin{figure}
%	\centering
%	\includegraphics[height=6cm,width=\linewidth]{../slotted_aloha_related_project/multiple_recepteur_capacity/pure_slot_throughput_mpr}
%	\caption{Spatial throughput with respect to normalized load}
%	\label{fig:pureslotthroughputmpr}
%\end{figure}
%
%Fig.~\ref{fig:pureslotthroughputmpr} illustrates the spatial throughput comparison. Due to page limitation, only macro diversity and best BS attach method are compared. In interval $\left[ 0, 1.0 \right]$, macro diversity for slotted ALOHA always has the best spatial through performance. In interval $\left[ 0, 0.2\right]$, macro diversity for pure ALOHA in worst case (i.e. with shadowing and maximum interference) still outperforms the best BS attach method.  In interval $\left[ 0.6, 1\right]$ where network packet loss rate is very high, the best BS attach method for slotted ALOHA with $8$ dB shadowing is better than macro diversity in pure ALOHA. In addition, shadowing has positive impact for best BS attach method when the normalized load is greater than $0.4$. 

%The diversity gain in terms of spatial throughput with $10\%$ packet loss rate constraint is the same with those in last three rows in Tab.~\ref{my-label}.
%In non-slotted ALOHA, the spatial throughput of macro reception diversity is 

%With normalized load, instead of reaching a maximal value then descending, we observe that the spatial throughput increases until a stable value. For the normalized load of interest in interval $\left[ 0, 0.3\right]$(zoomed part of Fig.~~\ref{fig:pureslotthroughputmpr}), we observe that, in slotted ALOHA, throughput of nearest BS attach method with shadowing effect $8$dB is  
%about $65\%$ ($0.118/0.18$ at normalized load $0.2$) of that in macro reception diversity case.
%In non-slotted ALOHA, throughput of the worst case is about $60\%$ of that in macro reception diversity. In addition, for macro reception diversity, the throughput ratio between non-slotted and slotted ALOHA is just $88\%$ at normalized load $0.2$.

%\begin{table}[]
%	\centering
%	\caption{Maximum normalized load under packet loss rate $10\%$ and corresponding diversity gain}
%	\label{my-label}
%	\resizebox{\columnwidth}{!}{%
%		\begin{tabular}{@{}lcccc@{}}
%			\toprule
%			& \multicolumn{1}{l}{Shadowing} & \multicolumn{1}{l}{Slotted ALOHA} & \multicolumn{1}{l}{\shortstack{Pure ALOHA\\Avg. Interference}} & \multicolumn{1}{l}{\shortstack{Pure ALOHA\\Max. Interference}} \\ \midrule
%			\multirow{2}{*}{Nearest BS Attach}                                   & 0dB                           & $0.05000$                         & $0.03756$                                                      & $0.02504$                                                      \\
%			& 8dB                           & $0.02333$                         & $0.01750$                                                      & $0.01166$                                                      \\ \midrule
%			\multirow{2}{*}{Best BS Attach}                                      & 0dB                           & $0.0500$                          & $0.03756$                                                      & $0.02504$                                                      \\
%			& 8dB                           & $0.03565$                         & $0.02674$                                                      & $0.01783$                                                      \\ \midrule
%			Macro Reception Diversity                                            & 0/8dB                         & $0.1957$                          & $0.1468$                                                       & $0.09787$                                                      \\ \midrule
%			\multirow{2}{*}{\shortstack{Diversity gain\\against Best BS attach}} & 0dB                           & \multicolumn{3}{c}{$\times3.91$}                                                                                                                                    \\
%			& 8dB                           & \multicolumn{3}{c}{$\times5.49$}                                                                                                                                    \\ \bottomrule
%		\end{tabular}%
%	}
%\end{table}

%that formula $(\ref{eq:bs_nst_att_analytical})$ is a monotonically increasing function of shadowing effect standard error $\sigma$, which means that shadowing effect has a negative impact. This is contrast with intuition that shadowing effect shall improve packet loss rate due to the power level diversity it brings. The reason of performance degradation is: shadowing effect is like to have the magic to ``change`` the distance between device and BS for each transmission. In other words, the geographical nearest BS is not always the best target to transmit packets if without accurate power  control to invert path loss.

%\begin{table}[]
%\centering
%\caption{Maximum normalized load under packet loss rate $10\%$ and corresponding diversity gain}
%\label{my-label}
%\resizebox{\columnwidth}{!}{%
%	\begin{tabular}{lcccc}
%		\hline
%		& \multicolumn{1}{l}{\shortstack{Shadowing}} & \multicolumn{1}{l}{\shortstack{Slotted\\ALOHA}} & \multicolumn{1}{l}{\shortstack{Pure ALOHA\\Average \\Interference}} & \multicolumn{1}{l}{\shortstack{Pure ALOHA\\Maximum \\Interference}} \\ \hline
%		\multirow{2}{*}{Best BS Attach}                                      & 0dB                           & $0.050$                           & $0.038$                                                        & $0.025$                                                        \\
%		& 8dB                           & $0.036$                           & $0.027$                                                        & $0.018$                                                        \\ \hline
%		\shortstack{\shortstack{Macro Reception\\Diversity} }                                            & 8dB                         & $0.20$                            & $0.15$                                                         & $0.1$                                                        \\ \hline
%		
%		\shortstack{\shortstack{Diversity gain\\against Best BS attach} }                                            & 8dB                         & $0.20$                            & $0.15$                                                         & $0.1$                                                        \\ \hline
%		
%		
%		\multirow{2}{*}{\shortstack{Diversity gain\\against Best BS attach}} &
%		0dB                           & \multicolumn{3}{c}{$\times4.0$}                                                                                                                                     \\
%		& 8dB                           & \multicolumn{3}{c}{$\times5.5$}                                                                                                                                     \\ \hline
%	\end{tabular}%
%}
%\end{table}

%Recall that for system with pure ALOHA and without error correction coding techniques, the analytical results are derived by using a simple upper bound. By simulation, we observe that the deviation with theoretical results is very slight (see Fig.~\ref{fig:subfig:c}), which confirms that the proposed simple-to-use upper bound is an good approximation.
