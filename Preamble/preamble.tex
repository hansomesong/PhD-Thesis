% ******************************************************************************
% ****************************** Custom Margin *********************************

% Add `custommargin' in the document class options to use this section
% Set {innerside margin / outerside margin / topmargin / bottom margin}  and
% other page dimensions
\ifsetCustomMargin
  \RequirePackage[left=37mm,right=30mm,top=35mm,bottom=30mm]{geometry}
  \setFancyHdr % To apply fancy header after geometry package is loaded
\fi

% Add spaces between paragraphs
%\setlength{\parskip}{0.5em}
% Ragged bottom avoids extra whitespaces between paragraphs
\raggedbottom
% To remove the excess top spacing for enumeration, list and description
%\usepackage{enumitem}
%\setlist[enumerate,itemize,description]{topsep=0em}

% *****************************************************************************
% ******************* Fonts (like different typewriter fonts etc.)*************

% Add `customfont' in the document class option to use this section

\ifsetCustomFont
  % Set your custom font here and use `customfont' in options. Leave empty to
  % load computer modern font (default LaTeX font).
  %\RequirePackage{helvet}

  % For use with XeLaTeX
  %  \setmainfont[
  %    Path              = ./libertine/opentype/,
  %    Extension         = .otf,
  %    UprightFont = LinLibertine_R,
  %    BoldFont = LinLibertine_RZ, % Linux Libertine O Regular Semibold
  %    ItalicFont = LinLibertine_RI,
  %    BoldItalicFont = LinLibertine_RZI, % Linux Libertine O Regular Semibold Italic
  %  ]
  %  {libertine}
  %  % load font from system font
  %  \newfontfamily\libertinesystemfont{Linux Libertine O}
\fi

% *****************************************************************************
% **************************** Custom Packages ********************************

% ************************* Algorithms and Pseudocode **************************

\usepackage{algpseudocode}


% ********************Captions and Hyperreferencing / URL **********************

% Captions: This makes captions of figures use a boldfaced small font.
%\RequirePackage[small,bf]{caption}

\RequirePackage[labelsep=space,tableposition=top]{caption}
\renewcommand{\figurename}{Fig.} %to support older versions of captions.sty


% *************************** Graphics and figures *****************************

%\usepackage{rotating}
%\usepackage{wrapfig}

% Uncomment the following two lines to force Latex to place the figure.
% Use [H] when including graphics. Note 'H' instead of 'h'
%\usepackage{float}
%\restylefloat{figure}

% Subcaption package is also available in the sty folder you can use that by
% uncommenting the following line
% This is for people stuck with older versions of texlive
%\usepackage{sty/caption/subcaption}
%\usepackage{subcaption}
\usepackage[tight,footnotesize]{subfigure}

% ********************************** Tables ************************************
\usepackage{booktabs} % For professional looking tables
\usepackage{multirow}
\usepackage{paralist}
\usepackage{longtable} % to support long table over 2 pages.
 %https://tex.stackexchange.com/questions/2441/how-to-add-a-forced-line-break-inside-a-table-cell
 \usepackage{makecell}

\let\labelindent\relax
\usepackage{enumitem}% http://ctan.org/pkg/enumitem

%\usepackage[
%nonumberlist, %do not show page numbers
%acronym,      %generate acronym listing   -> Not used in this example (see line with %%% )
%toc,          %show listings as entries in table of contents
%section]      %use section level for toc entries
%{glossaries}
\usepackage[nonumberlist,toc]{glossaries}
\makeglossaries %Activate glossary commands

%Term definitions
\newglossaryentry{mtc}{name=MTC, description={Machine Type Communication}}
\newglossaryentry{m2m}{name=M2M, description={Machine Type Communication}}
\newglossaryentry{htc}{name=HTC, description={Human Type Communication}}
%\newglossaryentry{3gpp}{name=3GPP, description={3rd Generation Partner Project}}
\newglossaryentry{lpwan}{name=LPWAN, description={Low Power Wide Area Network}}
\newglossaryentry{rnti}{name=RNTI, description={Radio Network Temporary Identifier}}
\newglossaryentry{prd-rnti}{name=PRD-RNTI, description={Periodic Radio Network Temporary Identifier}}
\newglossaryentry{tti}{name=TTI, description={Transmission Time Interval}}
\newglossaryentry{pdcp}{name=PDCP, description={Packet Data Convergence Protocol}}
\newglossaryentry{rrc}{name=RRC, description={Radio Resource Control}}
\newglossaryentry{rb}{name=RB, description={Resource Block}}
\newglossaryentry{pdf}{name=PDF, description={Probability Density Function}}
\newglossaryentry{cdf}{name=CDF, description={Cumulative Distribution Function}}
\newglossaryentry{lt}{name=LT, description={Laplace Transformation}}
\newglossaryentry{gsm}{name=GSM, description={Global System for Mobile communication, originally Group Special }}
%\usepackage{multicol}
%\usepackage{longtable}
%\usepackage{tabularx}


% ******************************* Math *********************************
\usepackage{amsmath,amssymb,amsthm} % For including math equations, theorems, symbols, etc
\usepackage{mathrsfs} %为了使用花体字母
\usepackage{mathtools}
\usepackage{breqn}
%to support indicator function,
%source:https://tex.stackexchange.com/questions/26637/how-do-you-get-mathbb1-to-work-characteristic-function-of-a-set
\usepackage{dsfont} 

\usepackage{hyperref} %to support clickable equation numeration
\usepackage{xcolor}\hypersetup{linkcolor=red, linkbordercolor=red}
\colorlet{linkequation}{green}
\newcommand*{\SavedEqref}{}
\let\SavedEqref\eqref
\renewcommand*{\eqref}[1]{%
	\begingroup
	\hypersetup{
		linkcolor=linkequation,
		linkbordercolor=linkequation,
	}%
	\SavedEqref{#1}%
	\endgroup
}

\colorlet{section}{red}
\newcommand*{\SavedRef}{}
\let\SavedRef\ref
\renewcommand*{\ref}[1]{%
	\begingroup
	\hypersetup{
		linkcolor=section,
		linkbordercolor=section,
	}%
	\SavedRef{#1}%
	\endgroup
}
%% Custom commands
\newcommand{\erf}{\mathrm{erf}\,}
\newcommand{\erfc}{\mathrm{erfc}\,}
\newcommand{\abs}[1]{\left| #1 \right|}
%% Custom commands, to define the mathmatical operator argmin
\newcommand{\argmax}[1]{\underset{#1}{\operatorname{arg}\,\operatorname{max}}\;}
\newcommand{\argmin}[1]{\underset{#1}{\operatorname{arg}\,\operatorname{min}}\;}

\DeclarePairedDelimiter\ceil{\lceil}{\rceil}
\DeclarePairedDelimiter\floor{\lfloor}{\rfloor}

% To support writing algorithm
% Excellent typsetting algorithm in chinese:
% http://www.zhixing123.cn/ubuntu/typeset-algorithm-in-latex.html
\usepackage{algorithm}
%\usepackage[noend]{algpseudocode}
\usepackage{algpseudocode}
\usepackage{algorithmicx}


\newtheorem{theorem}{Theorem}[section]
\newtheorem{corollary}{Corollary}[theorem]
\newtheorem{lemma}[theorem]{Lemma}
%\newtheorem{proof}[theorem]{Proof}
% *********************************** SI Units *********************************
\usepackage{siunitx} % use this package module for SI units


% ******************************* Line Spacing *********************************

% Choose linespacing as appropriate. Default is one-half line spacing as per the
% University guidelines

% \doublespacing
% \onehalfspacing
% \singlespacing


% ************************ Formatting / Footnote *******************************

% Don't break enumeration (etc.) across pages in an ugly manner (default 10000)
%\clubpenalty=500
%\widowpenalty=500

%\usepackage[perpage]{footmisc} %Range of footnote options


% *****************************************************************************
% *************************** Bibliography  and References ********************
% package "bibentry" is a part of natbib. We need this to support list of publications
\usepackage{natbib} 
\usepackage{bibentry}
\nobibliography*
%\usepackage{cleveref} %Referencing without need to explicitly state fig /table

% Add `custombib' in the document class option to use this section
\ifuseCustomBib
   \RequirePackage[square, sort, numbers, authoryear]{natbib} % CustomBib

% If you would like to use biblatex for your reference management, as opposed to the default `natbibpackage` pass the option `custombib` in the document class. Comment out the previous line to make sure you don't load the natbib package. Uncomment the following lines and specify the location of references.bib file

%\RequirePackage[backend=biber, style=numeric-comp, citestyle=numeric, sorting=nty, natbib=true]{biblatex}
%\bibliography{References/references} %Location of references.bib only for biblatex

\fi

% changes the default name `Bibliography` -> `References'
\renewcommand{\bibname}{References}


% ******************************************************************************
% ************************* User Defined Commands ******************************
% ******************************************************************************

% *********** To change the name of Table of Contents / LOF and LOT ************

%\renewcommand{\contentsname}{My Table of Contents}
%\renewcommand{\listfigurename}{My List of Figures}
%\renewcommand{\listtablename}{My List of Tables}


% ********************** TOC depth and numbering depth *************************

\setcounter{secnumdepth}{3}
\setcounter{tocdepth}{3}


% ******************************* Nomenclature *********************************

% To change the name of the Nomenclature section, uncomment the following line

%\renewcommand{\nomname}{Symbols}
\renewcommand{\nomname}{List of Abbreviations}

%\nomenclature[g-pi]{$\pi$}{ $\simeq 3.14\ldots$}
% ********************************* Appendix ***********************************

% The default value of both \appendixtocname and \appendixpagename is `Appendices'. These names can all be changed via:

%\renewcommand{\appendixtocname}{List of appendices}
%\renewcommand{\appendixname}{Appndx}

% *********************** Configure Draft Mode **********************************

% Uncomment to disable figures in `draft'
%\setkeys{Gin}{draft=true}  % set draft to false to enable figures in `draft'

% These options are active only during the draft mode
% Default text is "Draft"
%\SetDraftText{DRAFT}

% Default Watermark location is top. Location (top/bottom)
%\SetDraftWMPosition{bottom}

% Draft Version - default is v1.0
%\SetDraftVersion{v1.1}

% Draft Text grayscale value (should be between 0-black and 1-white)
% Default value is 0.75
%\SetDraftGrayScale{0.8}


% ******************************** Todo Notes **********************************
%% Uncomment the following lines to have todonotes.
\ifsetDraft
	\usepackage[colorinlistoftodos]{todonotes}
	\newcommand{\qs}[1]{\todo[author=Qipeng, size=\small,inline,color=green!40]{#1}}
	\newcommand{\qsong}[1]{\textcolor{blue}{(\textbf{[Qipeng's comment] }\textit{#1}})}
	\newcommand{\xl}[1]{\todo[author=Xavier, size=\small,inline,color=red!40]{#1}}
	\newcommand{\loutfi}[1]{\todo[author=Loutfi, size=\small,inline,color=blue!40]{#1}}
\else
	\newcommand{\mynote}[1]{}
	\newcommand{\qs}[1]{}
	\newcommand{\qsong}[1]{}
	\newcommand{\xl}[1]{}
	\newcommand{\loutfi}[1]{}
	\newcommand{\listoftodos}{}
\fi



% Example todo: \mynote{Hey! I have a note}
