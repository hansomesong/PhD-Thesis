\section{Introduction}
Machine-to-machine (M2M) communication, also called Machine type communication (MTC) in 3GPP's terminology, refers to an emerging technology which enables machine devices to be connected to a wide-area cellular network. This promising technology is expected to facilitate the creation of new applications and business models. In terms of carrier networks, due to the ubiquitous coverage, mature subscription management system, the 3GPP cellular networks are the best candidates for handling M2M traffic. As an example, it is estimated that over $2$ billion M2M devices will be connected to cellular networks in $2020$ \cite{Eri11}.

The integration of M2M into current cellular networks poses many challenges, since M2M traffic exhibits completely different characteristics from human-to-human (H2H) such as: a large number of short sessions, a number of battery operated devices, difficulty of replacing or recharging batteries for devices, and so on. In the past, mobile cellular networks were designed with objectives of maximizing spectral efficiency and minimizing latency. Nowadays, in the evolution from 4G to 5G, the differences between these two communication paradigms (H2H and M2M) should be considered and some design principles shifts are required.   

In the literature, lots of research efforts are made to mitigate the challenges imposed by the M2M traffic. The issue of Radio Access Network (RAN) overload control is identified as the first improvement area by 3GPP \cite{3GPP/TS/37868V11}, which leads to plenty of proposals for reducing random access collision and latency. Progressively, energy related issues (i.e., energy efficiency, power efficiency, etc.) are deemed as the key problems that determines if M2M communication is accepted as a promising communication technology \cite{lu11GRS}. To this end, diverse aspects are exploited by research community. Some researchers apply cooperative design such as clustering method \cite{YuanHo12}\cite{azari14}. Discontinuous reception (DRX) and Idle state in LTE/LTE-A are further optimized in \cite{Gupta2013}. Random access procedure is simplified by leveraging the transmission of small payload in \cite{ChenY10machine}. New radio resource allocation algorithms are proposed to achieve energy efficiency/saving \cite{AijazTNCA14}. Some researchers also consider to avoid random access procedure by leveraging the periodicity feature of M2M \cite{qipeng2015an}.
However, there are few works about the performance evaluation in terms of energy efficiency or power efficiency for frequently used multiple access strategies (e.g., CDMA/FDMA/TDMA). To our best knowledge, Dhillion et al.~\cite{Dhi13} made the first trial and proposed a system model to compare the multiple access mechanisms such as CDMA/FDMA/TDMA with metric of power efficiency and energy efficiency, but their work has not covered some factors important in real networks. For instance, the packet length is constant and the power control is ignored in \cite{Dhi13}. In this paper, we consider the aforementioned ignored factors and extend the system model in order to reevaluate the performance between CDMA and FDMA, and verify whether their obtained design guides are still valid in the extended system model.