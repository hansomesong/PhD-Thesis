\section{Conclusion}
\label{sec:conclusion} 
%In this paper, we extend a previously proposed system model by taking into account the impact of variable packet length and the imperfect power control, and reevaluate the energy efficiency and system capacity for two typical multiple access strategies: uncoordinated CDMA and coordinated FDMA. The metric selected for measuring power efficiency is the average transmit power for devices situated in the circle area covered by a single cell. The energy efficiency is measured by the product of time and average transmit power. The system capacity is the maximum supported base station load intensity. The conclusions obtained from the comparison between uncoordinated CDMA and coordinated FDMA are: although uncoordinated CDMA exhibits features such as simplicity and no signaling overhead, it is not always a suitable multiple access for future M2M-included cellular networks. The arguments are following. First, compared with coordinated FDMA, the power efficiency of uncoordinated CDMA performance decreases significantly when BS load intensity increases, especially for the situation where there exist various packet lengths for M2M devices. Second, the power control is very important for the performance of uncoordinated CDMA, but with imperfect power control (which is inevitable in a piratical system), CDMA surely requires a higher level of transmit power, especially when power control error variance is large. Therefore, coordinated access strategy, especially FDMA, can often be a better choice for the future cellular network.
%
%New version:
In this paper, we extend a previously proposed system model by taking into account the impact of variable packet length and imperfect power control, and evaluate the energy efficiency and system capacity for two typical multiple access strategies: uncoordinated CDMA and coordinated FDMA. The metric selected for measuring power efficiency is the average transmit power for devices situated in the circle area covered by a single cell. The energy efficiency is measured by the product of time and average transmit power. The system capacity is the maximum supported base station load intensity. 
The conclusions obtained from the comparison between uncoordinated CDMA and coordinated FDMA are: 
coordinated access strategy, especially FDMA, can often be a better choice for the future cellular network.
Although with power control error, uncoordinated CDMA is a considerable multiple access schema for cellular M2M network dedicated for small data transmission due to its simplicity and no signaling overhead. However, the power efficiency performance of uncoordinated CDMA degrades significantly when BS load intensity increases. Furthermore, the system capacity, namely the maximum supported BS load intensity is rather limited when power control error is obvious.

