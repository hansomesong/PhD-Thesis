%------------------------------------正文开始----------------------------------------------
\section{M2M Traffic Characterization}
\label{sec:overview-diff-traffic}
A comprehensive comparison between MTC and HTC is resumed in Table~\ref{tab:differences between M2M and H2H}.
The illustration of this difference helps to rethink the design of some principles and the optimization guidelines.
Some explanations about Table \ref{tab:differences between M2M and H2H}: 
\begin{itemize}[leftmargin=*, noitemsep]
	\item First, the representative device in H2H communication is smartphone, which is equipped with more and more computational capacity. The complexity of M2M devices is various: in application such as remote monitoring, it could be in format of sensor with transceiver and simplified processor. In Intelligent Transport System (ITS), it could be regarded as a smartphone without screen.
	\item Second, the experiment results in \cite{FirstLook12} reveal that compared to H2H communication, cellular M2M traffic suffers from a higher packet loss ratio, and the reason may be the adverse deployment location and the lack of UI (e.g., screen) to show the signal strength in its place. 
	\item Third, since at present, a majority of cellular M2M applications are based on GSM/UMTS technologies, MTC mainly supports SMS or data reporting service. We could imagine more innovation M2M services when 4G networks even 5G networks are largely rolled out.
\end{itemize}

Not all MTC applications have the same characteristics and not every optimization is suitable to all applications; therefore, features are defined to provide some structure to the customer and the network is then tuned accordingly to needs.

%shown in Tab.~{3GPP-MTC-feature} \cite{bartoli2011low}.
% "90% of the M2M devices across all applications are stationary" from Chapter 1.4.1.3 of book : M2M communications: a system approach
\begin{table}[]
	\centering
	\caption{Difference Between MTC and HTC}
	\label{tab:differences between M2M and H2H}
	\resizebox{\textwidth}{!}{%
		\begin{tabular}{@{}lll@{}}
			\toprule
			Item                                                                      & M2M                                                                                                                                                                                        & H2H                                                                                                                                                                             \\ \midrule
			Delay range                                                               & $10$ ms $\sim$ several minutes \cite{LienCL11}                                                                                                                                             & \begin{tabular}[c]{@{}l@{}}$250$ ms (voice) to few seconds \\ (email for example)\end{tabular}                                                                                  \\ \midrule
			Device composition                                                        & \begin{tabular}[c]{@{}l@{}}GSM/UMTS/LTE module,\\  extension slots, USB, memory, CPU, etc.\end{tabular}                                                                                    & \begin{tabular}[c]{@{}l@{}}GSM/UMTS/LTE/Wifi module\\ GPS, Bluetooth, USB, memory, CPU, \\ flash storage,etc.\end{tabular}                                                     \\ \midrule
			Packet loss radio                                                         & Relatively high \cite{FirstLook12}                                                                                                                                                         & Low                                                                                                                                                                             \\ \midrule
			Mobility                                                                  & \begin{tabular}[c]{@{}l@{}}Most of the M2M devices\\  (90\% according to \cite{opac-b1133822})\\  are stationary.\end{tabular}                                                             & \begin{tabular}[c]{@{}l@{}}Human are very rarely considered\\  fixed in practical mobile networks\end{tabular}                                                                  \\ \midrule
			Support service                                                           & Mainly SMS or data reporting                                                                                                                                                               & SMS/Voice/Web/Multimedia, etc.                                                                                                                                                  \\ \midrule
			\begin{tabular}[c]{@{}l@{}}Session duration\\ /frequency\end{tabular}     & \begin{tabular}[c]{@{}l@{}}Short but more or less frequent \cite{chou2014machine},\\  depending on the applications: \\ monitoring, transport or others\end{tabular}                       & Long but less frequent                                                                                                                                                          \\ \midrule
			Uplink                                                                    & MTC traffic is mainly generated in uplink                                                                                                                                                  & \begin{tabular}[c]{@{}l@{}}Traditionally less traffic in uplink \\ but increase rapidly with the flourishing of interactive\\  applications such as social network\end{tabular} \\ \midrule
			Downlink                                                                  & \begin{tabular}[c]{@{}l@{}}Less traffic except for some application \\ requiring interaction between sensors \\ and MTC servers, for example \\ consumer electronics use case\end{tabular} & \begin{tabular}[c]{@{}l@{}}Currently most traffic, for instance, \\ web browsing and multimedia\end{tabular}                                                                    \\ \midrule
			Message size                                                              & \begin{tabular}[c]{@{}l@{}}Generally very short. \\ In some cases could increase, \\ for example, if video sequences are uploaded\end{tabular}                                             & \begin{tabular}[c]{@{}l@{}}Typically big, especially for multimedia and\\  real-time transmission\end{tabular}                                                                  \\ \midrule
			Number of devices                                                          & Hundreds or thousands of devices per Base Station                                                                                                                                          & \begin{tabular}[c]{@{}l@{}}At most hundreds of UE, \\ typically tens of UEs per base station \cite{laya14}\end{tabular}                                                         \\ \midrule
			\begin{tabular}[c]{@{}l@{}}Battery life \\ requirement\end{tabular}       & \begin{tabular}[c]{@{}l@{}}Up to a few years, \\ especially for deployment locations \\ with difficult access\end{tabular}                                                                 & \begin{tabular}[c]{@{}l@{}}Order of days or weeks, \\ Human could easily recharge their device\end{tabular}                                                                     \\ \midrule
			\begin{tabular}[c]{@{}l@{}}Key metrics\\ for user experience\end{tabular} & Energy efficiency, latency                                                                                                                                                                 & Delay, throughput, packet loss                                                                                                                                                  \\ \bottomrule			
		\end{tabular}
	}
\end{table}