\section{Conclusion}
\label{sec:overview-conclusion}
The arrival of billions of connected machines in the short and mid term is a huge challenge for wireless networks, especially at the RAN level. LPWA technologies and cellular 3GPP solutions will be the main support used for M2M applications. In this chapter, we describe the present state of these technologies and the evolutions as expected today.

We propose a synthesis for the QoS demands and the difference of characteristics between H2H and M2M. We then review the proposals for radio coverage and service of these machines. We identify the advantage of cellular networks for this expected service. The 3GPP cellular networks have a mature infrastructure to provide wide coverage, high-availability service and user subscription/management system but the shortages of 3GPP networks are relatively high energy consumption level and cost of hardware with regard to LPWA networks. These challenges are addressed by some research proposals that we summarize in this article. In terms of LPWA network such as LoRaWAN, their significant advantage is their low energy consumption design and low-cost hardware. However, their disadvantage is that the operators should deploy dedicated infrastructure for providing LPWA-related service. 

According to our survey, to improve the device side energy efficiency in the future cellular networks, we get some design guidelines as follows: \begin{itemize}[leftmargin=*, noitemsep]
	\item The possible approaches to improve device side energy efficiency for cellular MTC include: cooperative relaying, design of energy efficient signaling and operation, radio resource allocation and packet scheduling strategies and Energy-efficient random access procedure and MAC;
	\item It is a better solution to employ cooperative relaying, since it can be combined with other emerging technologies such as D2D communication, ad hoc networks research results and LoRa technology, etc.
	\item The radio resource allocation and packet scheduling schemes allows to get energy efficiency while keeping a certain level of QoS, however it is difficult to design this kind of schemes simultaneously satisfying the QoS provisioning for both human and MTC users.  
	\item No matter by which approach, to gain energy efficiency is always with sacrifice of other system performances such as packet delay. Thus it is important to jointly use the aforementioned approaches, for example joint design of random access control and radio resource allocation, to seek for a trade off between energy efficiency and other system performances.
	\item It is always useful and necessary to rethink design of the radio access network. Thus, it is necessary to study the performance via analytical system models with mathematical such as stochastic geometry.
\end{itemize}