\section{M2M Related Standardization Efforts}
To address the difficulties for well supporting cellular MTC, the Standardization Developing Organizations (SDO) have launched their activities in their own fields. The dominant players in M2M landscape are ETSI M2M and 3GPP LTE. 3GPP is very active in M2M landscape. The focus of 3GPP is the improvement of radio access and core network to facilitate MTC over 3GPP networks. The first 3GPP report related to M2M~\cite{3GPP/TR/facilitating} issued in 2007 indicates a huge market potential for M2M beyond the current market segment. During the age of 3G, there have been little developments about MTC, since CDMA-based 3G systems are not suitable for low power operations. With the emergence of OFDM-based LTE, cellular M2M has become of interest and a set of further documents has been issued. MTC features and service requirements are defined in~\cite{3GPP/service-requirement}. An architectural reference model for MTC, key issues and possible solutions, are presented in~\cite{3GPP/TR/23888V11}.
%The contribution of Release 11 with regard to MTC... Release 12: Study network improvements for MTC device to MTC device communications via one or more PLMNs (direct-mode communication between devices is out of scope), etc. 
Recently, 3GPP also study to introduce a new class of User Equipment (UE) with low-cost feature in Release 13~\cite{3GPP/low-cost-device}.
In addition, 3GPP also aligns with ETSI Technical Committee M2M work.
ETSI M2M is composed by various manufacturers, operators and service providers, among others. To enable interoperability between M2M services and networks, ETSI established a Technical Committee (TC) in 2009 focusing on M2M service level and defined a M2M reference architecture and interfaces specification~\cite{ETSI/TS/102/690}.

%M2M-related Projects in industrial field
Besides, there are some international projects for facilitating MTC over 3GPP cellular networks. The EXALTED project (Expanding LTE for Devices, 2010-2013)~\cite{EXALTED}, a Europe FP7 project, is one of the flagship projects in M2M landscape. The main goal of this project is to develop a cost-, spectrum-, and energy-efficient radio access technology for M2M applications, the so-called LTE-M overlay, adapted to coexist within a high-capacity LTE network. Special attention was paid to scalability issues (to, e.g., avoid congestion in the random access procedures) and cost aspects, to ensure affordability of LTE M2M modules. 
Within this project, a lot of specific issues related to M2M communication in cellular network are studied, including radio resource allocation, relaying, security, PHYsical Layer, coding, emergency and rescue networks along with important standardization activity~\cite{bartoli2011low}\cite{GotsisLA12}. 
The METIS (Mobile and Wireless Communications Enablers for the Twenty-Twenty Information Society) is a Europe FP7 project from 2012 to 2015. Its global objective is to lay the foundation for the 5G system. With regard to MTC, they focus how to efficiently support Massive Machine-Type Communication (mMTC) and Utra-reliable Machine-Type Communication (uMTC) in the future 5G, by studying technologies about radio Links, multi-node/multi-antenna transmission and so on.
LOLA (Achieving Low-Latency in Wireless Communications)~\cite{lola} project focuses on physical and MAC layer techniques aimed at achieving low-latency transmission in cellular (LTE and LTE-A) and wireless mesh networks.