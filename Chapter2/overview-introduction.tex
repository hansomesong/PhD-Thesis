\section{Introduction}


With regard to cellular M2M related surveys, Taleb et al.~\cite{journals/cm/TalebK12} focus MTC devices subscription control and network congestion/overload control. Chen et al.~\cite{chen2014machine} talk research efforts for efficient MTC and explore various M2M-related issues such as deployment, operation, security and privacy. Andres et al.~\cite{laya14} make a survey of proposals improving the operation of random access channel of LTE/LTE-A and evaluate the energy consumption of LTE RACH procedure. Poncela et al. \cite{poncela2015m2m} identify the limitations of 4G for MTC (signaling, scheduler) and resume the improvements of LTE/LTE-A to handle M2M traffic. Several review papers~\cite{GhavimiF2015}\cite{mehmood2015mobile} discuss MTC in 3GPP LTE/LTE-A networks, introduce M2M uses cases in detail and identify the challenges with regard to M2M over LTE/LTE-A, e.g., random access congestion, resource allocation with QoS provisioning. Wang et al.~\cite{wang2012survey} survey and discuss various remarkable techniques, in terms of all components of the mobile networks (e.g., data centers, marcocell, femtocell, etc.), toward green mobile cellular networks. Ismail et al.~\cite{ismail2014survey} investigate the energy-efficiency from perspective of network operators and mobile users. Yang et al.~\cite{yang2015software} make a survey about software-defined wireless network (SDWN) and wireless network virtualization (WNV) for the future mobile wireless networks, which help define the future mobile wireless network architecture to tackle with heterogeneous traffic.

To our best knowledge, a comprehensive survey about device side energy efficiency issues in radio networks used for cellular M2M service is still not available in the literature. Therefore, the goal of this article is to compare and categorize existing M2M-related energy efficiency proposals before discussing the trends for cellular M2M research. In addition, in this article we want to provide a short overview of cellular M2M applications, detail different types of classification of M2M services and propose a synthesis for the QoS demands. We also review the advances of Low Power Wide Area networks, which are MTC-dedicated networks, and today are experiencing a rapid development. The rest of this article is organized as follows. Section~\ref{sec:m2m-app-class} presents the typical cellular M2M applications and several classifications according to different criteria, introduces a QoS requirement table for some typical cellular M2M applications. Section~\ref{sec:diff-traffic} compares the differences between H2H and M2M in terms of traffic characteristics. Section~\ref{sec:ref-m2m-arch} first talks about conventional M2M solutions in cellular networks then presents the advance of reference M2M network architecture. Section~\ref{sec:lpwa} resumes the development of Low Power Wide Area networks. Section~\ref{sec:proposals} presents, categorizes and compares all found proposals related to energy issues for MTC in cellular networks. Section~\ref{sec:conclusion} gives the conclusions obtained from this survey.