%*******************************************************************************
%*********************************** First Chapter *****************************
%*******************************************************************************

\chapter{Introduction}  %Title of the First Chapter

\ifpdf
    \graphicspath{{Chapter1/Figs/Raster/}{Chapter1/Figs/PDF/}{Chapter1/Figs/}}
\else
    \graphicspath{{Chapter1/Figs/Vector/}{Chapter1/Figs/}}
\fi


%********************************** %First Section  **************************************
\section{Motivation} %Section - 1.1 



%********************************** %Second Section  *************************************
\section{Background of this thesis} %Section - 1.2



%********************************** % Third Section  *************************************
\section{The Contributions}  %Section - 1.3 
\subsection{Survey about energy-efficiency for M2M-included wireless networks}
To cope with the low energy consumption level or high energy efficiency issue, which is a basic requirement for most of the cellular M2M applications, we have finished one survey which provides a global view of the network technologies previewed for cellular M2M. In this survey, we study the existing classifications of M2M applications according to different criteria in the literature. The comparison of traffic characteristics between M2M and human-to-human is also proposed. Quality of service (QoS) requirements for typical M2M applications are resumed. The advance of reference M2M network architectures proposed by the Standard Development Organization (SDO) is investigated. We identify two possible effort directions to improve the energy efficiency for cellular M2M. The first one is to evolve the current existing $3$rd Generation Partnership Project (3GPP) Consortium cellular networks to effectively support MTC (Machine Type Communication). The other direction is to design M2M-dedicated networks from scratch, which are often called low-power wide-area networks (LPWAN). We review, compare and categorize the proposals related to energy issues of cellular M2M mainly over the period from $2011$ to $2015$ for the first direction. We present the development of LPWAN networks for the other research directions. We conclude that the cooperative relaying, the design of energy-efficient signaling and operation, the new radio resource allocation schemes, and the energy-efficient random access procedure are the main points of improvement. It is important to jointly use the aforementioned approaches, for example, joint design of random access control and radio resource allocation, to seek for a trade-off between energy efficiency and other system

\subsection{Evaluation of multiple access strategies with power control error in M2M networks}
In this work, we extend the an existing research work by taking into account some important issues that were not yet addressed, mainly the existence of machines with different packet lengths and the effect of imperfect power control. With the proposed system model, we evaluate the power efficiency, energy efficiency and system capacity for uncoordinated CDMA and coordinated FDMA. Through numerical results, we find that coordinated FDMA is more resistant to various packet lengths of M2M devices packets in terms of power efficiency and is not influenced by imperfect power control. Thus coordinated multiple strategies, especially FDMA, are more suitable for the future M2M-included cellular networks and deserve further optimization works. With respect to uncoordinated CDMA, although its performance is affected by BS load intensity and power control, it is still a considerable choice due to its simplicity and no signaling overhead, when the BS load intensity is not high and the power control policy is suitable. 

\subsection{An analytical framework to study S-ALOHA performance used in LPWAN networks}
The S-ALOHA (i.e. slotted-ALOHA) protocol is recently regaining interest in Lower Power Wide Area Networks (LPWAN) handling M2M traffic. Despite intensive studies since the birth of S-ALOHA, the special features of M2M traffic and requirements highlight the importance of analytical models taking into account performance-affecting factors and giving a thorough performance evaluation. My contribution is to fulfill such a necessity: We jointly consider the impact of capture effect, diversity of transmit power levels with imperfect power control. We propose a low-complexity but still accurate analytical model capable of evaluating S-ALOHA in terms of packet loss rate, throughput, energy-efficiency and average number of transmissions. The proposed model is able to facilitate dimensioning and design of S-ALOHA based LPWAN. The comparison between simulation and analytical results confirms the accuracy of the proposed model. The design guides about S-ALOHA based LPWAN deduced from our model are: the imperfect power control can be positive with capture effect and appropriate transmit power diversity strategy. The transmit power diversity strategy should be determined by jointly considering network charges level, power control precision and SINR threshold to achieve optimal performance of S-ALOHA.

\subsection{Performance analysis of macro reception diversity in large-scale ALOHA Network}
In cellular networks, the packet is sent in unicast mode: the destination Base Station (BS) is indicated by the terminal. However, it also could be sent in broadcast mode, and benefit from macro reception diversity. The latter is a capability that a wireless network has,  allowing each device transmits messages in an omni-direction way without attach procedure. All the BS are the potential receivers of any sent message. Each BS autonomously demodulates and decodes the packets. The network is able to remove the duplicated received messages. 

Due to its advantages, BS reception diversity has been applied by some Lower Power Wide Area Networks, such as SigFox and LoRaWAN. We propose a tractable model to evaluation the gain brought by this capability, under fading and shadowing effect, in large wireless network with stochastic geometry. We get simple closed-form formulae for the packet loss rate, which were not known before. These formulae are very useful to analyze LPWAN networks and to quantify the gain brought by macro-diversity. Secondarily, we gathered several available results about the analysis of ALOHA in a short paper. This synthesis can be interesting for the scientific community. This work is submitted to IEEE communication letter and still under review.

\subsection{A periodic polling service in LTE network to support periodic M2M traffic}
% First, I study how to adjust currently deployed in large-scale LTE networks. 
LTE/LTE-A networks are designed to well meet the human-to-human (H2H) communication. Unfortunately, current LTE random access procedure is not able to efficiently handle MTC traffic. As a firs step, we analyze traditional random access procedure and identify the deficiency for M2M use case. In response to the found deficiency, we propose a polling service that avoids contention. This service is integrated into LTE access network, fully compatible with the standard access mechanism and able to manage a large range of polling periods (typically from \boldmath{$1$} minute to \boldmath{$28$} days).
This proposed service reduces the transmission overhead and thus improves the energy efficiency for MTC devices. It also reduces access network overload in radio access network by avoiding random access. 
Compared with traditional random access mechanism, numerical results show that with proposed service one eNodeB (eNB) can easily support up to \boldmath{$15000$} MTC devices without network access collision.  

\section{Structure of the Thesis}  %Section - 1.3 

