\section{Contributions and Organization of Manuscript}
\label{sec:introduction-contributions}
Efficiently supporting the huge number of MTC devices within wireless networks still remains as a great challenge, which is the focus of this thesis. From our studies, we try to answer the questions including but not limited to: For cellular networks, how to leverage the widely deployed LTE/LTE-A system to support energy efficiency constrain M2M application? For LWPAN system that has macro reception diversity\footnote{This term will be explained in detail in Chapter~\ref{chapter:ieee-cl-17} } but use non slotted ALOHA, such as Sigfox, what is the performance gain against slotted system? If the transmit power are various, what's the performance gain in slotted LPWAN? 
Which is the better choice for M2M systems? These issues are still not fully investigated in the literature. The contributions of this thesis can be summarized into five parts.
\subsection{A Survey about energy-efficiency for M2M-included wireless networks}
We first make a survey that provides a global view of the network technologies for M2M in wireless networks. In this survey, we study the classifications of M2M applications and compare traffic characteristics between M2M and H2H. Quality of service (QoS) requirements for typical M2M applications are resumed. The advances of reference M2M network architectures proposed by the Standard Development Organization (SDO) are investigated. As the main part of this survey,
we review, compare and categorize the proposals related to energy issues of cellular M2M mainly over the period from $2011$ to $2015$. We also present the development of LPWAN networks. Finally, we observe that the cooperative relaying, the design of energy-efficient signaling and operation, the new radio resource allocation schemes, and the energy-efficient random access procedure are the main points of improvement. It is important to jointly use the aforementioned approaches, for example, joint design of random access control and radio resource allocation, to seek for a trade-off between energy efficiency and other system. This part of work has been published in the journal of EURASIP~\cite{song2016survey}.

%\subsection{Evaluation of multiple access strategies with power control error in M2M networks}
%In this work, we extend the an existing research work by taking into account some important issues that were not yet addressed, mainly the existence of machines with different packet lengths and the effect of imperfect power control. With the proposed system model, we evaluate the power efficiency, energy efficiency and system capacity for uncoordinated CDMA and coordinated FDMA. Through numerical results, we find that coordinated FDMA is more resistant to various packet lengths of M2M devices packets in terms of power efficiency and is not influenced by imperfect power control. Thus coordinated multiple strategies, especially FDMA, are more suitable for the future M2M-included cellular networks and deserve further optimization works. With respect to uncoordinated CDMA, although its performance is affected by BS load intensity and power control, it is still a considerable choice due to its simplicity and no signaling overhead, when the BS load intensity is not high and the power control policy is suitable. 

\subsection{An analytical framework to study the performance of S-ALOHA in LPWAN networks}
The S-ALOHA (i.e. slotted-ALOHA) protocol is recently regaining interest in Low Power Wide Area Networks (LPWAN) handling M2M traffic. Despite intensive studies since the birth of S-ALOHA, the special features of M2M traffic and requirements highlight the importance of analytical models taking into account performance-affecting factors and giving a thorough performance evaluation. 

Our contribution is to fulfill such a necessity: We jointly consider the impact of capture effect, diversity of transmit power levels with imperfect power control. We propose a low-complexity but still accurate analytical model capable of evaluating S-ALOHA in terms of packet loss rate, throughput, energy-efficiency and average number of transmissions. The proposed model is able to facilitate dimensioning and design of S-ALOHA based LPWAN. The comparison between simulation and analytical results confirms the accuracy of the proposed model. 

The design guides about S-ALOHA based LPWAN deduced from our model are: the imperfect power control can have positive consequence with capture effect and appropriate transmit power diversity strategy. The transmit power diversity strategy should be determined by jointly considering network charges level, power control precision and SINR threshold to achieve optimal performance of S-ALOHA.

\subsection{Performance analysis of macro reception diversity in large-scale ALOHA Network}
In cellular networks, the packet is sent in unicast mode: the destination Base Station (BS) is indicated by the terminal. However, it also could be sent in broadcast mode, and benefit from macro reception diversity. The latter is a capability that a wireless network has,  allowing each device transmit messages in an omni-direction way without attach procedure. All the BS are the potential receivers of any sent message. Each BS autonomously demodulates and decodes the packets. The network is able to remove the duplicated received messages. 

Due to its advantages, BS reception diversity has been applied by some Lower Power Wide Area Networks, such as SigFox and LoRaWAN~\cite{ietf-lpwan-overview-03}. We propose a tractable model to evaluation the gain brought by this capability, under fading and shadowing effect, in large wireless network with stochastic geometry. We get simple closed-form formulas for the packet loss rate, which were not known before. These formulas are very useful to analyze LPWAN networks and to quantify the gain brought by macro-diversity. Secondarily, we gathered several available results about the analysis of ALOHA in a short paper. This synthesis can be interesting for the scientific community. This work has been published in IEEE communication letter~\cite{song2017evaluation}.

\subsection{A periodic polling service in LTE network to support periodic M2M traffic}
%TODO: Review this part once that I finish the RB consumption analysis.
The last contribution is one proposal in LTE/LTE-A networks to handle MTC traffic. It is known that LTE/LTE-A networks are designed to well meet the human-to-human (H2H) communication. Unfortunately, current LTE random access procedure is not able to efficiently handle MTC traffic. 

As a first step, we analyze traditional random access procedure and identify its inefficiency for M2M use case. We then propose a polling service that avoids contention. This service is integrated into LTE access network, fully compatible with the standard access mechanism and able to manage a large range of polling periods (typically from \boldmath{$1$} minute to \boldmath{$28$} days).
This proposed service reduces the transmission overhead and thus improves the energy efficiency for MTC devices. It also reduces access network overload in radio access network by avoiding random access. 

Traditional random access mechanism and our proposal are compared in terms of RNTI (Radio Network Temporary Identifier) and RB (Resource Block) consumption. The numerical results show that with the proposed service one eNodeB (eNB) can easily support up to \boldmath{$15000$} MTC devices without network access collision. In terms of RB consumption, our proposal avoids the $XX$ RB consumptions for the downlink direction. 